\documentclass[12pt]{article}
\usepackage[dvips]{attachfile2}
\usepackage{math}


\newcommand{\Fourier}[0]{\mathcal{F}}

\begin{document}

\topmargin=-1in

\title{Efficient Dealiasing of Fast Fourier-Transform Convolutions}
\author{John C. Bowman and Malcolm Roberts}
\maketitle

\section{1-D Dealiased Fast Fourier Transforms}
\subsection{$1/2$ Dealiased Data}
Suppose $N$ is a multiple of $2$. In terms of the $N$th primitive root of unity,
$$
\zeta_N\doteq \exp\(\frac{2 \pi i}{N}\),
$$
the discrete inverse Fourier transform may be written
$$
u_j=\sum_{k=0}^{N-1}\zeta_N^{jk} \hat u_k\qquad j=0,\ldots,N-1.
$$
Notice that $\zeta_N^2=\zeta_{N/2}$ and $\zeta_N^N=1$.

If $\hat u_k=0$ for $k \ge N/2$ then
\be
u_{2\ell}
=\ds\sum_{k=0}^{N/2-1}\zeta_{N/2}^{\ell k} \hat u_k
%=\ds\sum_{m=0}^{N/4-1}\zeta_{N/2}^{2\ell m} \hat u_{2m}+
%\zeta_N^{2\ell}\sum_{m=0}^{N/4-1}\zeta_{N/2}^{2\ell m} \hat u_{2m+1}.
\ee

Likewise,
\be
u_{2\ell+1}
=\ds\sum_{k=0}^{N/2-1}\zeta_{N/2}^{\ell k} \zeta_N^k\hat u_k
%=\ds\sum_{m=0}^{N/4-1}\zeta_{N/2}^{2\ell m} \zeta_N^{2m}\hat u_{2m}
%+\zeta_N^{2\ell+1}\sum_{m=0}^{N/4-1}\zeta_{N/2}^{2\ell m} \zeta_N^{2m}\hat u_{2m+1}
\ee

These terms need to be computed for $\ell=0,\ldots,N/2-1$.
So the scaling is $2\fr{N}{2}\log {N/2}=N\log(N/2)$.


The odd and even terms of the convolution can be separately computed,
multiplied term-by-term, and transformed back to Fourier space:
\bec
\hat u_k=\sum_{j=0}^{N-1}\zeta_N^{-kj} u_j
=\sum_{\ell=0}^{N/2-1}\zeta_N^{-2k\ell} u_{2\ell}
+\zeta_N^{-k}\sum_{\ell=0}^{N/2-1}\zeta_N^{-2k\ell} u_{2\ell+1}
=\sum_{\ell=0}^{N/2-1}\zeta_{N/2}^{-k\ell} u_{2\ell}
+\zeta_N^{-k}\sum_{\ell=0}^{N/2-1}\zeta_{N/2}^{-k\ell} u_{2\ell+1}
\qquad k=0,\ldots,\fr{N}{2}-1.
\ee

We can therefore compute the convolution with two 1024 blocks (rather than one
2048 block), to get a dealiased 1024 convolution (with the 2/4 rule).

\newpage
\subsection{$2/3$ Dealiased Data}
The convolution in the Navier--Stokes equations require a $2/3$ padding in
order to avoid problems with aliasing.  Suppose that the total number of modes
(including zero-padding) is $N=3m$.  The inverse discrete Fourier transform
is then
$$
u_j=\sum_{k=0}^{N-1}\zeta_N^{jk} \hat u_k
=\sum_{k=0}^{2m-1}\zeta_N^{jk} \hat u_k
$$
Then 
\bel
u_{3\ell}= \sum_{k=0}^{2m-1}\z_{3m}^{3\ell k} \hat u_k
=\sum_{k=0}^{m-1}\z_{m}^{\ell k} \hat u_k
+\sum_{k=m}^{2m-1}\z_{m}^{\ell k} \hat u_k
=\sum_{k=0}^{m-1}\z_{m}^{\ell k} \hat u_k
+\sum_{k=0}^{m-1}\z_{m}^{\ell (k+m)} \hat u_{k+m}
=\sum_{k=0}^{m-1}\z_{m}^{\ell k} \(\hat u_k+\hat u_{k+m}\).\label{DFTm}
\eel
Equation~\ref{DFTm} is a DFT of length $m$,
each requiring $m\log m$ operations. Following a similar procedure,
with $r=1$ or $r=2$,
\bel
\label{dfft23g}
u_{3\ell +r}\no=
 \sum_{k=0}^{m-1}\z_{m}^{\ell k} \(\z_N^{rk} \hat u_k + \z_N^{r(k+m)}\hat u_{k+m}\)
\eel
Thus, the total number of operations is $3 m \log m = N \log\frac{N}{3}$.

The forward Fourier transform appears as
\be
\hat u_k=\sum_{j=0}^{N-1}\zeta_N^{-kj} u_j
=\sum_{r=0}^{2}\zeta_N^{-rk}\sum_{\ell=0}^{m-1}\zeta_N^{-3\ell k} u_{3\ell+r}
=\sum_{r=0}^{2}\zeta_N^{-rk}\sum_{\ell=0}^{m-1}\zeta_m^{-\ell k} u_{3\ell+r}
\qquad k\no =0,\ldots,pm-1.
\ee

\newpage
\subsubsection{1D Complex to Real Transforms of 2/3 Dealiased Data}

Consider the cryptic comment
\begin{quotation}
  ``Also, this will work with complex to real transforms without modification,
  since $\zeta_N^{-k}=\zeta_N^k{}^*$.''
\end{quotation}
As it turns out, $2/3$ dealiased data works particularly well with the
Hermiticity condition, $\hat{u}_{-k}=\hat{u}^*_k$, which guarantees
that the $x$-space data is real-valued. Consider a
wavenumber-truncated vector ranging from $k=-(m-1)$ to $k=m-1$, which
has $2m-1$ elements.

The backwards (complex-to-real) transform becomes, on letting $k'=m+k$,
\bec
u_{3\ell +r}\no=\sum_{k=-m+1}^{m-1}\z_{m}^{\ell k} \z_N^{rk} \hat u_k
=\sum_{k'=1}^{m-1}\z_{m}^{\ell k'} \z_N^{r(k'-m)} \hat u_{k'-m}
+\sum_{k=0}^{m-1}\z_{m}^{\ell k} \z_N^{rk} \hat u_k
=\sum_{k=1}^{m-1}\z_{m}^{\ell k} \z_N^{-r(m-k)} \hat u_{m-k}^*
+\sum_{k=0}^{m-1}\z_{m}^{\ell k} \z_N^{rk} \hat u_k
=\sum_{k=0}^{m-1}\z_{m}^{\ell k} \(w_{k,r}+w_{m-k,r}^*\),
\ee
where
$$
w_{k,r}=\z_N^{rk} \(\hat u_k-\half \hat u_0\d_{k,0}\).
$$
The forwards transform becomes
\be
\hat u_k=\sum_{j=0}^{N-1}\zeta_N^{-kj} u_j
=\sum_{r=-1}^{1}\zeta_N^{-rk}\sum_{\ell=0}^{m-1}\zeta_N^{-3\ell k} u_{3\ell+r}
=\sum_{r=-1}^{1}\zeta_N^{-rk}\sum_{\ell=0}^{m-1}\zeta_m^{-\ell k} u_{3\ell+r}
\qquad k\no =0,\ldots,m-1.
\ee

Note: without the hermiticity condition, the backwards transform is
\bec
u_{3\ell +r}\no=\sum_{k=-m+1}^{m-1}\z_{m}^{\ell k} \z_N^{rk} \hat u_k
=\sum_{k'=1}^{m-1}\z_{m}^{\ell k'} \z_N^{r(k'-m)} \hat u_{k'-m}
+\sum_{k=0}^{m-1}\z_{m}^{\ell k} \z_N^{rk} \hat u_k
=\sum_{k=0}^{m-1}\z_{m}^{\ell k} w_{k,r},
\ee
where
$$
w_{k,r}=
\cases{
\hat u_0&if $k=0$,\cr
\z_N^{rk}(\hat u_k+\z_3^{-r}\hat u_{k-m})&if $1\le k\le m-1$.\cr
}
$$


\subsubsection{2D Complex to Real Transforms of 2/3 Dealiased Data}
The Hermiticity condition is now $\hat{u}_{-k,-\ell}=\hat{u}^*_{k,\ell}$.
The procedure is analagous as in 1D,
where we interpret each index as a vector, add dot products,
split the two-dimensional sum into lower and upper halves, and use the
change of variables $\vk=(m_x,m_y)+\vk'$.
\begin{comment}
\bec
u_{3u+r,3v+s}\no=\sum_{k=-m-1}^{m-1}\sum_{\ell=-m-1}^{m-1}
\z_{m}^{u k} \z_{m}^{v \ell} \z_N^{rk} \z_N^{s\ell} \hat u_{k,\ell}
=\sum_{k=-m-1}^{m-1}\z_{m}^{u k}\z_N^{rk}  
\sum_{\ell=1}^{m-1} \z_{m}^{v \ell} \z_N^{s(\ell'-m)} \hat u_{k,\ell'-m}
+\sum_{k=-m-1}^{m-1}\z_{m}^{u k}\z_N^{rk}
\sum_{\ell=0}^{m-1} \z_{m}^{v \ell} \z_N^{s\ell} \hat u_{k,\ell}
=\sum_{k=-m-1}^{m-1}\z_{m}^{-u k}\z_N^{-rk}  
\sum_{\ell=1}^{m-1} \z_{m}^{v \ell} \z_N^{-s(m-\ell')} \hat u_{k,m-\ell'}^*
+\sum_{k=-m-1}^{m-1}\z_{m}^{u k}\z_N^{rk}
\sum_{\ell=0}^{m-1} \z_{m}^{v \ell} \z_N^{s\ell} \hat u_{k,\ell}
\ee
\end{comment}


\newpage
\subsection{$p/q$ Dealiased Data}

In general, consider a ``$p/q$'' padding in which $N=qm$, and only $pm$ modes
are non-zero, with $p$ and $q$ relatively prime. Consider $u_{q\ell+r}$, with
$\ell=0 \dots m-1$, $r=0 \dots q-1$.
Then
\be
u_{q\ell+r} = \sum_{k=0}^{N-1}\z_N^{(q\ell+r)k} \hat u_k
= \sum_{k=0}^{pm-1}\z_{qm}^{(q\ell+r)k} \hat u_k
= \sum_{k=0}^{pm-1}\z_{m}^{\ell k}\z_{N}^{rk} \hat u_k
= \sum_{\a=0}^{p-1} \sum_{k=\a m}^{(\a+1)m-1}\z_{m}^{\ell k}\z_{N}^{rk} \hat u_k
= \sum_{\a=0}^{p-1} \sum_{\k=0}^{m-1}\z_{m}^{\ell (\k+\a m)}\z_{N}^{r(\k+\a m)}
\hat u_{\k+\a m}
%=  \sum_{\a=0}^{p-1}\z_N^{r \a m} \sum_{\k=0}^{m-1}\z_m^{\ell\k}\z_N^{r\k}\hat u_{\k+\a m}
=  \sum_{\k=0}^{m-1}\z_m^{\ell\k}\sum_{\a=0}^{p-1}\z_N^{r(\k+\a m)} \hat u_{\k+\a m}
\ee .
Note that, in the last line, we have the choice of which $\z$ to use, since
$\z_q^{r \a}=\z_N^{r \a m}$. Since there are $m$ choices of $r$ and $q$ choices
for $\ell$, we are left with $mq=N$ FFTs of length $p$, leaving on the order
of $N \log p = N \log (N/q)$ operations.  Again, while the computational
savings is only marginal, this formulation affords signficant improvements
in memory use and parallelizability.

The forward Fourier transform appears as
\be
\hat u_k=\sum_{j=0}^{N-1}\zeta_N^{-kj} u_j
=\sum_{r=0}^{q-1}\zeta_N^{-rk}\sum_{\ell=0}^{m-1}\zeta_N^{-q\ell k} u_{q\ell+r}
=\sum_{r=0}^{q-1}\zeta_N^{-rk}\sum_{\ell=0}^{m-1}\zeta_m^{-\ell k} u_{q\ell+r}
\qquad k\no =0,\ldots,pm-1.
\ee

%TODO: I think that this can be generalized in the following fashion:
%if $m$ of the $N$ modes are non-zero, let $p=N/\gcd(m,N)$. Then we
%have can divide up the DFT into $p$ DFTs of length $m$. Must work out
%the details.




\section{2D, 2/3-dealiased transforms}
Our goal is to apply this to dealiased pseudospectral simulations. If we 
avoid moving the $k$-space origin to the middle of the array, the Fourier
data is sparse, with a square of length $2/3$ being non-zero. Suppose that 
the first FFT is done in the $x$ direction.  Then, $1/3$ of the transforms
are zero, and can be ignored. The remaining transforms are $2/3$ dealiased,
each of which can be done with $n \log (2 n/3)$ operations. This leaves
the upper $1/3$ modes set to zero.  The final transform, done in the 
$y$-direction, takes $n \log(2 n/3)$ operations.  See figure \ref{dealias2d}.
The total cost is then
\be
\frac{2 n}{3} \log \frac{2 n}{3} +  n \log \frac{2 n}{3}
=\frac{5n}{3} \log \frac{2 n}{3},
\ee
using the dealiased FFT, as opposed to $2 n \log n$ for the naive FFT. This 
is $1/6$ faster, while using significantly less memory. Moreover, the
intermediary, $1/3$-sparse buffer can be kept allocated as an intermediary
buffer for all necessary transformations.
\begin{figure}[htbp]
  \begin{center}
    \includegraphics{dealias2d}
    \caption{Procedure for transforming 2D dealiased arrays.}
    \label{dealias2d}
  \end{center}
\end{figure}


\end{document}
