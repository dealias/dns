\documentclass[12pt,showpacs,showkeys,%preprint,
amsfonts,amsmath,
onecolumn,
floatfix,aps,superscriptaddress]{revtex4}
\usepackage{amsfonts,amsmath,array}
\usepackage{graphicx}
\usepackage{bm}
\usepackage{mal}
\usepackage{math}
\usepackage{cancel}
\begin{document}

\title{Pseudo-Spectral simulation of the Navier--Stokes equations}
\section{Purpose of this document}
These are notes regarding the implimentation of the DNS code.

\section{Two dimensional Navier--Stokes}
The vorticity formulation of the 2-D Navier--Stokes equations is
\begin{eqnarray}
  \frac{\partial \omega_k}{\partial t} 
  + \nu_k \omega_k 
  &=& \int dp \int dq \frac{\e_{kpq}}{q^2}
  \omega_p^* \omega_q^*
  +  F_k
  \\
  \epsilon_{kpq} &=& \( \hat z \cdot  p \times  q \)
  \delta \( k +  p +  q \)
\end{eqnarray}
where $ F_{ k}$ represents an external force. The 2D equations conserve
energy $E = \sum_i |u_i|^2$ and enstrophy $  Z = \sum_i k_i^2|u_i|^2$.

\section{The nonlinearity}

\subsection{Fourier Transforms}
In order to avoid aliasing errors due to quadratic source terms, the NS eqns 
must be padded.  That is, if we wish to evolve $u_k$ with $k = 0\dots, N-1$,
then we must pad the domain to include $u_k, k \in N, \dots, N + N/2$ directly
before any Fourier transform.  This is known as the ``$2/3$'' rule, in that 
only $2/3$ of the domain is actually used, with the remaining set to zero by
padding.

Let $\hat u_k$ be the Fourer transform of $u_j(x)$, and suppose that we instead
have a $2/4$ rule, with half the domain used for padding, as is the case for 
higher-order nonlinear terms. With padding, $u_k=0$ for $k > \frac{N}{2}$. 
Let $\zeta = e^{\frac{2 i \pi}{N}}$, i.e.\ an $N$th root of unity.  Ignoring the
normalization factor of $1/N$,
\begin{eqnarray}
u_j &=& \sum_{k=0}^{2N-1}\zeta ^{jk} \hat u_k 
\\
&=& \sum_{k=0}^{N-1}\zeta ^{jk} \hat u_k 
+\sum_{k=N}^{2N-1}\zeta ^{jk} \hat u_k 
\\
&=& \sum_{k=0}^{N-1}\zeta ^{jk} \hat u_k 
+\sum_{\ell=0}^{N-1}\zeta ^{j(\ell + N)} \hat u_{\ell+N} 
\\
&=& \sum_{k=0}^{N-1}\zeta ^{jk} \hat u_k 
+(-1)^j\sum_{\ell=0}^{N-1}\zeta ^{j\ell} \hat u_{\ell+N} 
,\qquad \text{since $\zeta^N=-1$}
\\
&=& \sum_{k=0}^{N-1}\zeta ^{jk} \hat u_k 
+(-1)^j\sum_{\ell=0}^{N-1}\zeta ^{j\ell} (-1)^{\ell+N}\hat u_\ell 
,\qquad \text{by the shift theorem, eq \eqref{shiftthm}}
\\
\label{speedy}
&=& \sum_{k=0}^{N-1}\zeta ^{jk} \(1 + (-1)^{j+k}\) \hat u_k 
\end{eqnarray}
I'm worried that we're using the shift theorem wrong here.  
However, it looks like we can eliminate half of the modes; either
the even ones (if $j$ is odd) or the odd ones (if $j$ is even) which
are cancelled by the $\(1 + (-1)^{j+k}\)$.  This works particularly well
with FFTs, which separate interlaced modes.  Which would be a really dramatic
increase in speed (more than I can believe right now, frankly.)

Note that equation \eqref{speedy} is not technically a FFT, since the summation
should go from $0$ to $2N-1$. However, the only modification needed is an
adjustment of the summation limit (and on each subsequent step.)


\begin{theorem}
This shift theorem (lifted from Wikipedia) is:
If $\mathcal{F}(\{x_n\})_k=X_k$
then
\begin{eqnarray}
  \mathcal{F}(\{ x_n\cdot e^{\frac{2\pi i}{N}n m} \})_k=X_{k-m}
\end{eqnarray}
and 
\begin{eqnarray}
  \label{shiftthm}
  \mathcal{F}(\{x_{n-m}\})_k=X_k\cdot e^{-\frac{2\pi i}{N}k m}.
\end{eqnarray}
\end{theorem}

asdf
\end{document}
