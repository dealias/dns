\documentclass[12pt,showpacs,showkeys,%preprint,
amsfonts,amsmath,
onecolumn,
floatfix,aps,superscriptaddress]{revtex4}
\usepackage{amsfonts,amsmath,array}
\usepackage{graphicx}
\usepackage{bm}
\usepackage{mal}
\usepackage{math}
\usepackage{cancel}
\begin{document}

\title{Pseudo-Spectral simulation of the Navier--Stokes equations}
\section{Purpose of this document}
These are notes regarding the implimentation of the DNS code.

\section{Two dimensional Navier--Stokes}
The vorticity formulation of the 2-D Navier--Stokes equations is
\begin{eqnarray}
  \frac{\partial \omega_{\v k}}{\partial t} 
  + \nu_{\v k} \omega_{\v k} 
  &=& \int dp \int dq \frac{\epsilon_{\v{kpq}}}{\v q^2}
  \omega_{\v p}^* \omega_{\v q}^*
  + \v F_{\v k}
  \\
  \epsilon_{\v{kpq}} &=& \(\hat{\v z} \dot \v p \times \v q \)
  \delta\(\v k + \v p + \v q \)
\end{eqnarray}
where $\v F_{\v k}$ represents an external force. The 2D equations conserve
energy $E = \sum_i |u_i|^2$ and enstrophy $  Z = \sum_i k_i^2|u_i|^2$.

\section{The nonlinearity}
In order to avoid aliasing errors due to quadratic source terms, the NS eqns 
must be padded.  That is, if we wish to evolve $u_k$ with $k = 0\dots, N-1$,
then we must pad the domain to include $u_k, k \in N, \dots, N + N/2$ directly
before any Fourier transform.  This is known as the ``$2/3$'' rule, in that 
only $2/3$ of the domain is actually used, with the remaining set to zero by
padding.

Let $\hat u_k$ be the Fourer transform of $u_j(x)$, and suppose that we instead
have a $2/4$ rule, with half the domain used for padding, as is the case for 
higher-order nonlinear terms. With padding, $u_k=0$ for $k > \frac{N}{2}$. 
Let $\zeta = e^{\frac{2 i \pi}{N}}$, i.e.\ an $N$th root of unity.  Ignoring 
normalization factors,
\bee
u_j &=& \sum_{k=0}^{2N-1}\zeta ^{jk} \hat u_k , j=0,\dots,N-1
\\
u_j &=& \sum_{k=0}^{2N-1}\zeta ^{jk} \hat u_k , j=N,\dots,2N-1
\eee

\bee
u_j &=& \sum_{k=0}^{2N-1}\zeta ^{jk} \hat u_k 
=\sum_{k=0}^{N-1}\zeta ^{jk} \hat u_k
\\
&=& \sum_{m=0}^{\frac{N}{2}-1} \zeta^{2mj}  u_{2m}
+\zeta^j\sum_{m=0}^{\frac{N}{2}-1}\zeta^{2mj} u_{2m+1}.
\eee
However, the second sum can be gained using the shift theorem (from wikipedia):
\be
\mathcal{F}(\{x_n\})_k=X_k \implies
\mathcal{F}(\{ x_n\cdot e^{\frac{2\pi i}{N}n m} \})_k=X_{k-m},
\ee
since
\be
u_{2m+1}
\ee



\end{document}
