\documentclass{amsart}
 
\usepackage{amsmath}
\usepackage{amssymb}
\usepackage{amsfonts}

\title{The Multispectral Method}
\author{Malcolm Roberts$^1$
\and
John C. Bowman$^1$
\and
Bruno Eckhardt$^2$
}
\address{$^1$Department of Mathematical and Statistical Sciences,
University of Alberta, Edmonton, Alberta T6G 2G1, Canada
 ({\tt mroberts@math.ualberta.ca}, {\tt bowman@math.ualberta.ca})}
\address{$^2$Fachbereich Physik, Philipps-Universit\"at Marburg, 
D-35032 Marburg, Germany\linebreak[4]
 ({\tt bruno.eckhardt@physik.uni-marburg.de})}
\date{}

\keywords{}

\subjclass{}

\begin{document}

\begin{abstract}

Spectral simulations of high Reynolds-number turbulence require a very
large number of active modes, ranging from the largest scale of the
system to the dissipation scale. Since there is no intermediate range
of quiescent modes, numerical techniques cannot rely on a separation
of scales to reduce the stiffness of the system. 

The \emph{multispectral method} uses spectral reduction 
[Bowman, Shadwick, and Morrison,  \emph{Phys.\ Rev.\ Lett.\ }{\bf 83}, 
5491 (1999)] to evolve time-dependent PDEs on a hierarchy of decimated
grids in Fourier space.  We can choose to decimate the grids keeping
all low-wavenumber modes intact, allowing us to simulate very small scales 
with far fewer degrees of freedom than an equivalent pseudo-spectral 
simulation.

In [Roberts, M.\ 2006.\ \emph{A Multi-Spectral Decimation Scheme for 
Turbulence Simulations}.\ Master's thesis, University of Alberta],
we demonstrated this technique using shell models of
turbulence.  Here, we apply this technique to two-dimensional Navier--Stokes 
turbulence, taking care that projection and prolongation between grids
conserve important quadratic invariants.

\end{abstract}

\maketitle

\end{document}
