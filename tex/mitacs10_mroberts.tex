\documentclass{amsart}
 
\usepackage{amsmath}
\usepackage{amssymb}
\usepackage{amsfonts}

\title{The Multispectral Method}
\author{Malcolm Roberts$^1$
\and
John C. Bowman$^1$
\and
Bruno Eckhardt$^2$
}
\address{$^1$Department of Mathematical and Statistical Sciences,
University of Alberta, Edmonton, Alberta T6G 2G1, Canada
 ({\tt mroberts@math.ualberta.ca}, {\tt bowman@math.ualberta.ca})}
\address{$^2$Fachbereich Physik, Philipps-Universit\"at Marburg, 
D-35032 Marburg, Germany\linebreak[4]
 ({\tt bruno.eckhardt@physik.uni-marburg.de})}
\date{}

\keywords{}

\subjclass{}

\begin{document}

\begin{abstract}

Spectral simulations of high Reynolds-number turbulence require a very
large number of active modes, ranging from the largest scale of the
system to the dissipation scale. Since there is no intermediate range
of quiescent modes, numerical techniques cannot rely on a separation
of scales to reduce the stiffness of the system. 

The \emph{multispectral method} generalizes spectral reduction 
[Bowman, Shadwick, and Morrison,  \emph{Phys.\ Rev.\ Lett.\ }{\bf 83}, 
5491 (1999)] to evolve time-dependent PDEs on a hierarchy of decimated
grids in Fourier space.  The grids can be decimated so that low-wavenumber
modes are preserved. High-Reynolds number turbulence can then be simulated
using far fewer degrees of freedom than required for a full pseudospectral
simulation.

In previous work [Roberts, \emph{A Multi-Spectral Decimation Scheme for
  Turbulence Simulations}, Master's thesis, University of Alberta (2006)],
the multispectral technique was demonstrated for shell models. 
Here, we apply the technique to the two-dimensional incompressible
Navier--Stokes equation, taking care that the projection and prolongation
operators between the grids conserve both energy and enstrophy. 
The nonlinear advection term is handled using efficient algorithms 
that we have recently developed for computing implicitly dealiased
convolutions ({\tt http://fftwpp.sourceforge.net}).

\end{abstract}

\maketitle

\end{document}

% LocalWords:  Multi dealiasing
