\documentclass[12pt,showpacs,showkeys,%preprint,
amsfonts,amsmath,
onecolumn,
floatfix,aps,superscriptaddress]{revtex4}
\usepackage{amsfonts,amsmath,array}
\usepackage{graphicx}
\usepackage{bm}
\usepackage{mal}
\usepackage{math}
\usepackage{cancel}
\begin{document}

\title{A Faster Fast Fourier-Transform for Zero-Padded Input}

Let 
\begin{eqnarray}
  \zeta_N = \exp\(\frac{2 \pi i}{N}\)
\end{eqnarray}
This is a discrete Fourier transform:
\begin{eqnarray}
  u_j = \sum_{k=0}^{N-1}\zeta_N^{jk} u_k
  =\sum_{k=0}^{N-1}e^{2 \pi ijk/N} u_k.
\end{eqnarray}
This is not a discrete Fourier transform:
\begin{eqnarray}
  \label{nodft}
  \sum_{k=0}^{N-1}e^{\frac{\pi ijk}{N}} u_k
  &=& \sum_{k=0}^{N/2-1}e^{\frac{2\pi ijk}{N}} u_{2k}
  +e^{\frac{\pi ij}{N}}\sum_{k=0}^{N/2-1}e^{\frac{2\pi ijk}{N}} u_{2k+1}
\end{eqnarray}
Wait, maybe that doesn't go anywhere. Ok, consider 2/4 padding, so
\begin{eqnarray}
  u_j&=& \sum_{k=0}^{N-1} e^{\frac{2 \pi ijk}{N}}u_k \\
  &=&  \sum_{k=0}^{N/2-1} e^{\frac{2 \pi ijk}{N}}u_k\\
  &=&
  \begin{cases}
    \sum_{k=0}^{N/2-1} e^{\frac{2 \pi i2\ell k}{N}}u_k, \quad j=2\ell\\
    \sum_{k=0}^{N/2-1} e^{\frac{2 \pi i(2\ell +1)k}{N}}u_k, \quad j=2\ell+1\\
  \end{cases}\\
  &=&
  \begin{cases}
    \sum_{k=0}^{N/2-1} e^{\frac{4 \pi i\ell k}{N}}u_k, \quad j=2\ell\\
    \sum_{k=0}^{N/2-1} e^{\frac{2\pi i k}{N}}e^{\frac{4 \pi i\ell k}{N}}u_k, \quad j=2\ell+1\\
  \end{cases}
  \label{ffts}
\end{eqnarray}
Now, equation \eqref{ffts} consists of two terms which are both
Fourier transforms with $N/2$ elements, one of which has a
modification of the source which depends on $k$. Each transform takes
$\frac{N}{2}\log \frac{N}{2}$ operations, so we are again left with $N
\log\frac{N}{2}$ scaling. We can calculate $u_{2\ell},
\ell=0,\dots,N/4$ and $u_{2\ell+1}, \ell=0,\dots,N/4$ using separate
(conventional) FFTs, and interlace them afterwards. While the speed saving
is only in the $\log$ term, the FFT can be done with half the memory (and
in parallel) which should be worthwhile.

\end{document}
