\documentclass[12pt]{article}

\title{\huge{\textbf{\\Two-Dimensional {\tt ProtoDNS} Code}}}
\author{Noel Murasko, Pedram Emami, and John C. Bowman\\University of Alberta\\Edmonton,
  Alberta, Canada}
\date{\today}

% PAGE LAYOUT
\setlength{\oddsidemargin}{0pt}
\setlength{\evensidemargin}{0pt}
\setlength{\marginparwidth}{0pt}
\setlength{\marginparpush}{0pt}
\setlength{\marginparsep}{0pt}
\setlength{\textwidth}{464pt}
\addtolength{\topmargin}{-2.0cm}
\addtolength{\textheight}{4.0cm}

\usepackage{bm}
\usepackage{mathrsfs}
\usepackage{amsmath}
\usepackage{amsfonts}
\usepackage{graphicx}
\usepackage{booktabs}
\usepackage{asymptote}
\usepackage{mathdef}
\usepackage{hyperref}
\usepackage{colordvi,color}
\def\v{\bm}
\usepackage{hypercitebracket}
%\usepackage{showlabels}
\usepackage{makeidx}

% MACRO DEFINITIONS
\let\divsign\div
\def\div{\grad\dot}
\def\lap{{\nabla}^2}
\def\lapinv{{\nabla}^{-2}}
\def\vw{{\v\w}}
\def\hyperindex#1{\index{#1}\hypertarget{#1}}
\def\In#1{\hyperindex{#1}{\emph{#1}}}
\def\Eq#1{(\ref{#1})}

% PACKAGE OPTIONS
\definecolor{heavyred}{cmyk}{0,1,1,0.25}
\definecolor{heavyblue}{cmyk}{1,1,0,0.25}
\hypersetup{
  pdftitle=,
  pdfpagemode=UseOutlines,
  citebordercolor=0 0 1,
  colorlinks=true,
  allcolors=heavyred,
  breaklinks=true,
  pdfauthor=,
  pdfpagetransition=Dissolve,
}
\hyphenation{protodns}

\bibliographystyle{rmp2}

\makeindex

\begin{document}
\maketitle

\thispagestyle{empty}
\begin{figure}[h]
\centering
\includegraphics{uofa}
\end{figure}
\newpage
\thispagestyle{empty}
\begin{center}
\ \vspace{20cm}\\
Noel Murasko, Pedram Emami, and John C. Bowman\\
ALL RIGHTS RESERVED\\
Reproduction of these notes in any form, in whole or in part, is permitted only for nonprofit educational use.
\end{center}
\newpage

\section{Introduction}
{\tt Protodns} is a pseudospectral code for
direct numerical simulation (DNS) of two-dimensional (2D) incompressible homogeneous
  turbulent flow with periodic boundary conditions in Fourier space. We
  will explain the set of governing equations and the way through which we
  can obtain the most numerically efficient known representation.
Here we must mention that as it can be inferred from the name of the code, it is the simpler version of the most efficient and complete DNS code for simulation of two-dimensional incompressible homogeneous turbulent flows with periodic boundary conditions in Fourier space called 2D code. So the main reason of having {\tt protodns} is essentially for educational purposes and so it does not exploit many possible implementation optimizations to speed up the simulation process. The reader who is interested in the most advanced, efficient, and complete version of this code, can refer to the 2D code available at \url{https://github.com/dealias/dns/tree/master/2d}.
%
\section{Governing Equations}
We start our work with the set of governing equations for incompressible
turbulent flows. The main set of governing equations are the
Navier--Stokes equation for momentum and the incompressibilty
condition for the velocity $\vu$,
\begin{equation}
\begin{cases}
\dfrac{\partial \vu}{\partial t} + \vu\cdot\grad\vu = -\grad P + \nu\lap\vu ,\\
\\
\div \vu=0.\label{GE}
\end{cases}
\end{equation}
We assume that the there is no external forcing on the system. The equation describing the evolution of the \In{vorticity}
$\vw=\curl\vu$ is sometimes more convenient:
$$\begin{cases}
\dfrac{\partial{\vw}}{\partial t} + (\vu\dot\grad)\vw=(\vw\dot\grad)\vu + \nu\lap\vw,\\
\\
\div \vu=0.
\end{cases}$$
This is obtained using the identity $\vu\cdot \grad\vu= \frac{1}{2}\grad u^2 - \grad\times(\grad\times \vu)$. One advantage of the vorticity equation is that it does not contain a pressure term. 
Moreover, in 2D, one can exploit the fact that the
vorticity vector is always parallel to the normal $\zhat$ of the plane of motion. 
The \In{vortex stretching} term is then given by $(\vw\dot\grad)\vu = 0$. The vorticity equation in 2D is then given by:
\begin{equation}\label{vorticity}
\begin{cases}
\dfrac{\partial{\vw}}{\partial t} + (\vu\dot\grad)\vw=(\vw\dot\grad)\vu + \nu\lap\vw,\\
\\
\div \vu=0.
\end{cases}
\end{equation}
Since the flow is incompressible, we can express
$\vu=\curl\vA$ in the Coulomb gauge $\div\vA=0$, so that
$$
\vw=\w\zhat=\curl(\curl \vA)=\grad(\div\vA)-\del^2 \vA=-\del^2 \vA.
$$
Hence $\del^2 A_x=\del^2 A_y=0$. Given periodic 
conditions one may then without loss of generality take $A_x=A_y=0$ so that
$\vA$, like $\vw$, has only one component $\w$, in the $\zhat$ direction.
It is conventional to define $\psi\doteq -A_z$ to be
the \In{stream function}, so that
\begin{equation}\label{u-psi}
\vu=\curl(-\psi \zhat)=\hat{\vz}\cross\grad\psi=-
  \frac{\partial\psi}{\partial y}\hat{\vx}+\frac{\partial\psi}{\partial x}\hat{\vy}
\end{equation}
and
\begin{equation}\label{omega}
\vw=(\lap\psi)\hat{\vz}.
\end{equation}

So using \Eq{u-psi} and \Eq{omega}, we can represent the velocity
vector with respect to the stream function as
$\vu=\hat{\vz}\cross\grad(\lapinv\omega)$.
Because the vorticity is always perpendicular to the plane of motion,
the term $(\vw\dot\grad)\vu$ on the right-hand side of the vorticity equation vanishes:
$$\frac{\partial \omega}{\partial t} + (\hat{\vz}\cross\grad(\lapinv\omega)\dot\grad)\omega= \nu\lap\omega .$$
The discrete Fourier transform ${\cal F}$ of the vorticity sampled at collocation
points $\vw_\vj\doteq\vw(\vx_\vj)$ over a two-dimensional lattice
$G$ of size $N\times N$,
$$
\vw_{\vk}=\sum_{\vj\in G} \vw_\vj e^{-\fr{2\pi i\vk\dot\vj}{N^2}}.
$$
evolves according to
\begin{equation}\label{Fourier}
\frac{\partial\omega_{\vk}}{\partial t} + {\cal F}_\vk\{\vu\cdot\grad\omega\}= -\nu k^2\omega_{\vk},
\end{equation}
where the Fourier transform of the advection term
$\vu\cdot\grad\omega$ can be written as a convolution:
\begin{equation}\label{Fourier-expand}
\frac{\partial\omega_{\vk}}{\partial t} + \sum_{\vp}{\frac{(\hat{\vz}\cross\vp)\dot\vk}{p^2}\omega_{\vp}\omega_{\vk-\vp}}= -\nu k^2\omega_{\vk}.
\end{equation}
\section{Reducing the Number of FFTs}
Instead of computing the advection term in Fourier space, we first transform to physical space using an inverse Fast Fourier Transform (FFT). Now, the nonlinear terms can computed by pointwise multiplication and addition. These terms are then transformed back into Fourier space with a forward FFT.

To obtain the advection term in \Eq{Fourier}, we take the $z$ component of the curl of this term:
\begin{equation}\label{zAdvection}
\vu\cdot\grad\omega =\hat{\vz}\cdot \left[\grad \times \left(\vu\cdot\grad\vu \right)\right]
\end{equation}
The most computationally expensive part of this algorithm are the number of FFTs. Naively inspecting the advection term,
\begin{equation}
\vu\cdot\grad\omega = u_1\frac{\partial \omega}{\partial x_1}+u_2\frac{\partial \omega}{\partial x_2},
\end{equation}
one might think that five FFTs are required at each time step: four
inverse FFTs to transfrom $u_1$, $ u_2$, $\partial \omega/\partial x_1$, and $\partial \omega/\partial x_2$ from Fourier space to physical space and one forward FFT to transform $u_1\left(\partial \omega/\partial x_1\right)+u_2\left(\partial \omega/\partial x_2\right)$ from physical space to Fourier space. If we rewrite the term in a clever way, however, we can get down to four FFTs.

To reduce the amount of writing, we write $\vu\cdot\grad\vu$ in Einstein notation (sum over repeated indices)
$$\vu\cdot\grad\vu =u_j\dfrac{\partial u_i}{\partial x_j}.$$
Note that with the incompressibility condition, $\dfrac{\partial u_j}{\partial x_j} =0$, this equation can be written as 
\begin{equation}\label{incompAdvection}
u_j\dfrac{\partial u_i}{\partial x_j} = \dfrac{\partial (u_iu_j)}{\partial x_j}.
\end{equation}
Now expanding \Eq{zAdvection} using \Eq{incompAdvection}, we have
\begin{align*}
\vu\cdot\grad\omega =\hat{\vz}\cdot \left[\grad \times \left(\vu\cdot\grad\vu \right)\right]&= \dfrac{\partial }{\partial x_1}\dfrac{\partial (u_2u_j)}{\partial x_j}  - \dfrac{\partial }{\partial x_2}\dfrac{\partial (u_1u_j)}{\partial x_j},\\
&= \dfrac{\partial }{\partial x_1}\left(\dfrac{\partial (u_2u_1)}{\partial x_1}+\dfrac{\partial u_2^2}{\partial x_2}\right)  - \dfrac{\partial }{\partial x_2}\left(\dfrac{\partial u_1^2}{\partial x_1}+\dfrac{\partial (u_1u_2)}{\partial x_2}\right),\\
&= \left(\frac{\partial^2}{\partial x_1^2} - \frac{\partial^2}{\partial x_2^2}\right)(u_1u_2) +\frac{\partial^2}{\partial x_1\partial x_2}\left(u_2^2-u_1^2\right).\\
\end{align*}
Thus, we can write \Eq{Fourier} as:
\begin{equation}\label{evolutionary}
  \frac{\partial\omega_{\vk}}{\partial t} =
  (k_x^2-k_y^2) {\cal F}_\vk\{uv\}+k_xk_y{\cal F}_\vk\{v^2-u^2\} -\nu k^2\omega_{\vk}.
\end{equation}
Two inverse transforms are required to compute $u$ and $v$ in physical
space, from which the quantities $uv$ and $v^2-u^2$ can then be
calculated and transformed back to Fourier space with two additional
forward transforms. The advection term in 2D can thus be calculated with four $\text{FFTs}$.
\section{The Energy, Enstrophy, and Palinstrophy} 
In this code, we compute three quantities from the simulation data: the energy $E$, the enstrophy $Z$, and palinstrophy $P$.  These quantities are only used in our analysis of the simulation. They are not inputs in any part of the algorithm. 
\subsection{Definitions}
The energy $E$ is a familiar concept in physics. In physical space, it is given by the mean squared velocity of the system:
$$
E\doteq \frac{1}{2} \int |\vu|^2.
$$
The enstrophy $Z$ is defined in terms of the rate of change of the energy. For an unforced and incompressible fluid, it can be shown \cite{frisch95} that
$$
\frac{\partial E}{\partial t} = -\nu \int |\vw|^2.
$$
We define the enstrophy as
$$
Z\doteq\frac{1}{2}\int |\vw|^2,
$$
so we have 
$$
\frac{\partial E}{\partial t} = -2\nu Z.
$$
Similarly, it can be shown \cite{frisch95} that rate of change of the enstrophy of a unforced, and incompressible fluid is given by
$$
\frac{\partial Z}{\partial t} = -\nu \int |\grad\times \vw|^2.
$$
The palinstrophy $P$ is then defined as
$$
P\doteq\frac{1}{2}\int |\grad \times \vw|^2,
$$
so we have 
$$
\frac{\partial Z}{\partial t} = -2\nu P.
$$
While these definitions are motivating, they do not help with computation. These definitions are in physical space, not Fourier space. Furthermore, the energy is in terms of the velocity, while in our code, we only evolve the vorticity. In the next subsections, we will express $E$, $Z$, and $P$ in terms of $\vw_\vk$. 
\subsection{The Energy}
In physical space, energy is defined as the mean squared velocity of the system. First, we must express the velocity in terms of the vorticity. Taking the curl of the vorticity gives us
$$\grad \times \vw = \grad \times(\grad \times \vu) = \grad(\grad\cdot \vu)- {\nabla}^2 \vu = - {\nabla}^2 \vu,$$
where we've used $\grad\cdot \vu = 0$. Now taking the Fourier transform of this equation and rearranging gives us:
$$\vu_\vk = i \frac{\vk\times \vw_\vk}{k^2}. $$
We also note that $\vk$, $\vu_\vk$, and $\vw_\vk$ are all mutually orthogonal. Now the energy is given by:
\begin{align*}
E &= \sum_{\vk} |\vw_\vk|^2 = \sum_{\vk} \left|i \frac{\vk\times \vw_\vk}{k^2} \right|^2,\\
& =\sum_{\vk}\frac{1}{k^4}(\vk\cdot \vk ) (\vw_\vk\cdot \vw_\vk)  , \\
& = \sum_{\vk}\frac{|\vw_\vk|^2}{k^2}.
\end{align*}
So we have:
$$E = \sum_{\vk}\frac{|\vw_\vk|^2}{k^2}.$$
\subsection{The Enstrophy}
In physical space, the enstrophy of the system is given by the mean squared vorticity of the system. To compute $Z$, we sum up the squared vorticity $|\vw_\vk|^2$ for each wave vector:
$$ Z = \sum_{\vk} |\vw_\vk|^2.$$
\subsection{The Palinstrophy}
In physical space, the palinstrophy is given by the mean squared curl of the vorticity. First, the Fourier transform of the curl of the vorticity is given by:
$$\cal{F}_{\vk}(\grad\times \vw) = i\vk\times \vw_\vk.$$
The palinstrophy is then given by
$$P = \sum_{\vk} |i\vk\times \vw_\vk|^2 = \sum_{\vk} (\vk\cdot \vk ) (\vw\cdot \vw)   = \sum_{\vk} k^2|\vw_\vk|^2 .$$
So we have: 
$$P = \sum_{\vk} k^2|\vw_\vk|^2 .$$
\section{Numerical simulation}
\subsection{The domain of simulation}
We notice that the velocity field in physical space is real. Because of this, we have the following relation:
\begin{equation}
\vu_\vk = \vu_{-\vk}^*,
\end{equation}
where we use $*$ to denote the complex conjugate. We exploit this to reduce the number of modes we have to compute explicitly. 
We take the domain of our solution to be: 
$$
\vk = (k_x, k_y): k_x \in [-m_x + 1, m_x -1],\ k_y \in 
\begin{cases}
[1, m_y -1]&\text{if }k_x \leq 0\\
[0, m_y -1]&\text{if }k_x > 0\\
\end{cases}.
$$
This is shown in as a diagram in Figure~(\ref{domain}). 

One might notice that we do not compute the velocity at the mode $\vk = (0,0)$. This isn't a problem, as this mode doesn't evolve in time. It corresponds to the mean flow of the system. For simplicity, we assert that the system has zero mean flow, by setting $u_{(0,0)} = 0$.
\begin{figure}[ht]
\begin{center}
\asyinclude{domain}
\caption{Domain of simulation}\label{domain}
\end{center}
\end{figure}
\newpage
\subsection{Solution algorithm}
We can now summarize the steps of the algorithm:
\begin{enumerate}
\item Initialize $\omega_{\vk}$ in Fourier space;
\item Calculate the velocity components $u_\vk$ and $v_\vk$ from $\omega_{\vk}$;
\item Take the inverse discrete fast Fourier transform
  ($\text{FFT}^{-1}$) of $u_\vk$ and $\u_\vk$;
\item Calculate $uv$ and $v^2-u^2$ term in physical space;
\item Take the forward discrete fast Fourier transform ($\text{FFT}$) of
$uv$ and $v^2-u^2$;
\item Update the values of $\omega_\vk$ by marching one step in time;
\item \emph{Optionally}, compute the energy, enstrophy, and palinstrophy from $\omega_\vk$.
\item Repeat steps 2 -- 7 for the desired number of time steps.
\end{enumerate}
A simplified version of the numerical algorithm is shown as a flow chart in Figure~(\ref{algorithm}). 

\begin{figure}[ht]
\begin{center}
\asyinclude{algorithm}
\caption{Solution algorithm}\label{algorithm}
\end{center}
\end{figure}

\hypertarget{Bibliography}{}
\pdfbookmark{Bibliography}{Bibliography}
\bibliography{refs}

%\hypertarget{Index}{}
%\pdfbookmark{Index}{Index}
%\printindex


\end{document}
