\documentclass[12pt]{article}

\title{\huge{\textbf{\\Two-Dimensional {\tt ProtoDNS} Code}}}
\author{Pedram Emami and John C. Bowman\\University of Alberta\\Edmonton,
  Alberta, Canada}
\date{\today}

% PAGE LAYOUT
\setlength{\oddsidemargin}{0pt}
\setlength{\evensidemargin}{0pt}
\setlength{\marginparwidth}{0pt}
\setlength{\marginparpush}{0pt}
\setlength{\marginparsep}{0pt}
\setlength{\textwidth}{464pt}
\addtolength{\topmargin}{-2.0cm}
\addtolength{\textheight}{4.0cm}

\usepackage{bm}
\usepackage{mathrsfs}
\usepackage{amsmath}
\usepackage{amsfonts}
\usepackage{graphicx}
\usepackage{booktabs}
\usepackage{asymptote}
\usepackage{mathdef}
\usepackage{hyperref}
\usepackage{colordvi,color}
\def\v{\bm}
\usepackage{hypercitebracket}
%\usepackage{showlabels}
\usepackage{makeidx}

% MACRO DEFINITIONS
\let\divsign\div
\def\div{\grad\dot}
\def\lap{{\nabla}^2}
\def\lapinv{{\nabla}^{-2}}
\def\vw{{\v\w}}
\def\hyperindex#1{\index{#1}\hypertarget{#1}}
\def\In#1{\hyperindex{#1}{\emph{#1}}}
\def\Eq#1{(\ref{#1})}

% PACKAGE OPTIONS
\definecolor{heavyred}{cmyk}{0,1,1,0.25}
\definecolor{heavyblue}{cmyk}{1,1,0,0.25}
\hypersetup{
  pdftitle=,
  pdfpagemode=UseOutlines,
  citebordercolor=0 0 1,
  colorlinks=true,
  allcolors=heavyred,
  breaklinks=true,
  pdfauthor=,
  pdfpagetransition=Dissolve,
}
\hyphenation{protodns}

\bibliographystyle{rmp2}

\makeindex

\begin{document}
\maketitle

\thispagestyle{empty}
\begin{figure}[h]
\centering
\includegraphics{uofa}
\end{figure}
\newpage
\thispagestyle{empty}
\begin{center}
\ \vspace{20cm}\\
Pedram Emami and John C. Bowman\\
ALL RIGHTS RESERVED\\
Reproduction of these notes in any form, in whole or in part, is permitted only for nonprofit educational use.
\end{center}
\newpage

\section{Introduction}
{\tt Protodns} is a pseudospectral code for
direct numerical simulation (DNS) of two-dimensional (2D) incompressible homogeneous
  turbulent flow with periodic boundary conditions in Fourier space. We
  will explain the set of governing equations and the way through which we
  can obtain the most numerically efficient known representation.
Here we must mention that as it can be inferred from the name of the code, it is the simpler version of the most efficient and complete DNS code for simulation of two-dimensional incompressible homogeneous turbulent flows with periodic boundary conditions in Fourier space called 2D code. So the main reason of having {\tt protodns} is essentially for educational purposes and so it does not exploit many possible implementation optimizations to speed up the simulation process. The reader who is interested in the most advanced, efficient, and complete version of this code, can refer to the 2D code available at \url{https://github.com/dealias/dns/tree/master/2d}.
%
\section{Governing Equations}
We start our work with the set of governing equations for incompressible
turbulent flows. The main set of governing equations are the
Navier--Stokes equation for momentum and the incompressibilty
condition for the velocity $\vu$,
\begin{equation}
\begin{cases}
\dfrac{\partial \vu}{\partial t} + \vu\cdot\grad\vu = -\grad P + \nu\lap\vu + \vF,\\
\\
\div \vu=0.\label{GE}
\end{cases}
\end{equation}
The equation describing the evolution of the \In{vorticity}
$\vw=\curl\vu$ is sometimes more convenient:
\begin{equation}\label{vorticity}
\begin{cases}
\dfrac{\partial{\vw}}{\partial t} + (\vu\dot\grad)\vw=(\vw\dot\grad)\vu + \nu\lap\vw + \curl\vF,\\
\\
\div \vu=0.
\end{cases}
\end{equation}
One advantage of the vorticity equation is that it does not contain
a pressure term.
Moreover in 2D, one can exploit the fact that the
vorticity vector is always parallel to the normal $\zhat$ of the plane of motion.
Since the flow is incompressible, we can express
$\vu=\curl\vA$ in the Coulomb gauge $\div\vA=0$, so that
$$
\vw=\w\zhat=\curl(\curl \vA)=\grad(\div\vA)-\del^2 \vA=-\del^2 \vA.
$$
Hence $\del^2 A_x=\del^2 A_y=0$. Given periodic 
conditions one may then without loss of generality take $A_x=A_y=0$ so that
$\vA$, like $\vw$, has only one component $\w$, in the $\zhat$ direction.
It is conventional to define $\psi\doteq -A_z$ to be
the \In{stream function}, so that
\begin{equation}\label{u-psi}
\vu=\curl(-\psi \zhat)=\hat{\vz}\cross\grad\psi=
  \frac{\partial\psi}{\partial y}\hat{\vx}+\frac{\partial\psi}{\partial x}\hat{\vy}
\end{equation}
and
\begin{equation}\label{omega}
\vw=(\lap\psi)\hat{\vz}.
\end{equation}

So using \Eq{u-psi} and \Eq{omega}, we can represent the velocity
vector with respect to the stream function as
$\vu=\hat{\vz}\cross\grad(\lapinv\omega)$.
Because the vorticity is always perpendicular to the plane of motion,
the term $(\vw\dot\grad)\vu$ on the right-hand side of the vorticity equation vanishes:
\begin{equation}\label{2D-vorticity-expand}
\frac{\partial \omega}{\partial t} + (\hat{\vz}\cross\grad(\lapinv\omega)\dot\grad)\omega= \nu\lap\omega + f.
\end{equation}
The discrete Fourier transform ${\cal F}$ of the vorticity sampled at collocation
points $\vw_\vj\doteq\vw(\vx_\vj)$ over a two-dimensional lattice
$G$ of size $N\times N$,
$$
\vw_{\vk}=\sum_{\vj\in G} \vw_\vj e^{-\fr{2\pi i\vk\dot\vj}{N^2}}.
$$
evolves according to
\begin{equation}\label{Fourier}
\frac{\partial\omega_{\vk}}{\partial t} + {\cal F}_\vk\{\vu\cdot\grad\omega\}= -\nu k^2\omega_{\vk} + f_{\vk}.
\end{equation}
where 
$$
\vf_{\vk}=\sum_{\vj\in G} \vf_\vj e^{-\fr{2\pi i\vk\dot\vj}{N^2}}
$$
and $\vf_\vj\doteq\vf(\vx_\vj)$,
where the Fourier transform of the advection term
$\vu\cdot\grad\omega$ can be written as a convolution:
\begin{equation}\label{Fourier-expand}
\frac{\partial\omega_{\vk}}{\partial t} + \sum_{\vp}{\frac{(\hat{\vz}\cross\vp)\dot\vk}{p^2}\omega_{\vp}\omega_{\vk-\vp}}= -\nu k^2\omega_{\vk} + f_{\vk},
\end{equation}

\section{Numerical simulation}
\subsection{The domain of simulation}
Now that we have our governing \Eq{Fourier} in Fourier space, then we need to take a numerical scheme to solve that equation. So at first we need to characterize our domain of simulation. As we mentioned before, because our intention is direct numerical simulation of homogeneous isotropic turbulence, so we have to avoid having walls or in fact we have to stay far enough from walls. This can be easily done by considering periodic boundary conditions. Considering this point, our schematic domain is shown in Figure~\ref{domain}. 
\begin{figure}[ht]
\begin{center}
\asyinclude{domain}
\caption{Domain of simulation}\label{domain}
\end{center}
\end{figure}
\subsection{Solution algorithm}
Now, we are to represent the numerical scheme that we have to take for solving our governing equation. 
Equation~\Eq{Fourier} can be written as an evolutionary equation for $\omega_\vk$ by taking all the terms except the time-derivative of $\omega$ to the right-hand side:
\begin{equation}\label{evolutionary}
  \frac{\partial\omega_{\vk}}{\partial t} =
  (k_x^2-k_y^2) {\cal F}_\vk\{uv\}+k_xk_y{\cal F}_\vk\{v^2-u^2\} -\nu k^2\omega_{\vk} + f_{\vk}.
\end{equation}

\pagebreak[4]
The numerical solution steps are:
\begin{enumerate}
\item Initialize $\omega_{\vk}$ in Fourier space;
\item Calculate the velocity components $u_\vk$ and $v_\vk$ from $\omega_{\vk}$;
\item Take the inverse discrete fast Fourier transform
  ($\text{FFT}^{-1}$) of $u_\vk$ and $\u_\vk$;
\item Calculate $uv$ and $v^2-u^2$ term in physical space;
\item Take the forward discrete fast Fourier transform ($\text{FFT}$) of
$uv$ and $v^2-u^2$;
\item Update the values of $\omega_\vk$ by marching one step in time;
\item Calculate the values of energy, enstrophy, and palinstrophy;
\item Go to Step 2.
\end{enumerate}
The flowchart of the above numerical algorithm is shown in Figure~\ref{algorithm}.
\begin{figure}[ht]
\begin{center}
\asyinclude{algorithm}
\caption{Solution algorithm}\label{algorithm}
\end{center}
\end{figure}
\subsection{2D case}
Two inverse transforms are required to compute $u$ and $v$ in physical
space, from which the quantities $uv$ and $v^2-u^2$ can then be
calculated and transformed back to Fourier space with two additional
forward transforms. The advective term in 2D can thus be calculated with four $\text{FFTs}$.

\hypertarget{Bibliography}{}
\pdfbookmark{Bibliography}{Bibliography}
\bibliography{refs}

\hypertarget{Index}{}
\pdfbookmark{Index}{Index}
\printindex


\end{document}
