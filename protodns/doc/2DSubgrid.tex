\documentclass[11pt,a4paper]{article}
\usepackage{amsmath}
\usepackage{amsfonts}
\usepackage{mathrsfs} 
\usepackage{cancel}
\usepackage[colorlinks]{hyperref}
\hypersetup{
    colorlinks=true,
    citecolor = blue,
    linkcolor = red
}
\usepackage{amssymb}
\usepackage{graphicx}
\usepackage[left=2cm,right=2cm,top=2cm,bottom=2cm]{geometry}
\newcommand{\eq}[1]{(\ref{#1})}
\author{Noel Murasko}
\title{Two-Dimensional Large Eddy Simulation}
\date{\today}
\begin{document}
\maketitle
\noindent 
\section{Filtering Equations}
Recall that direct numerical simulation of the `small scales' is too computationally expensive. In Large Eddy Simulations, aim to simulate only the large scales. This can be done formally by filtering our equations \cite{Leonard1974}. Consider some function $f(\boldsymbol{x}, t)$. The ``large scales'' of $f$ can be obtained by a convolution with a filter function $G$:
$$\overline{f} = \int G(\boldsymbol{x} - \boldsymbol{x}')f(\boldsymbol{x}')\ d\boldsymbol{x}'.$$
The filter function $G$ has an associated length scale $\Delta$. Scales smaller than $\Delta$ are removed in $\overline{f}$. 

Assuming the fluid is incompressible, we can express the Navier--Stokes equation and continuity equation in terms of their components:
$$\frac{\partial u_i}{\partial t} - \frac{\partial }{\partial x_j}(u_iu_j)+\frac{\partial P}{\partial x_i}- \nu \frac{\partial^2u_i}{\partial x_j^2} = F_i,$$
$$\frac{\partial u_i}{\partial x_i}  =0.$$
Here, $P$ is the pressure, $u_i$ is the velocity in the $i$th direction, and $F_i$ is the external force in the $i$th direction. Note that we use Einstein notation and sum over repeated indices. Applying a filter to these equations gives us:
$$\frac{\partial \overline{u}_i}{\partial t} - \frac{\partial }{\partial x_j}\left(\overline{u_iu_j}\right)+\frac{\partial \overline{P}}{\partial x_i}- \nu \frac{\partial^2\overline{u}_i}{\partial x_j^2}= \overline{F}_i,$$
$$\frac{\partial \overline{u}_i}{\partial x_i}  =0.$$
The difficulty of this equation is the nonlinear term $\left(\overline{u_iu_j}\right)$ which is the filtered product of the unfiltered velocities. We can't compute this term directly; we need a model for it.

To formulate such a model, we define the residual stress tensor as the difference between the filtered product of velocities and the product of filtered velocities \cite{Pope2000}:
$$\tau_{ij}^R := \overline{u_iu_j} - \overline{u}_i\overline{u}_j.$$
$\tau_{ij}^R$ is called the \emph{residual stress tensor}. Now we can write the non-linear term as:
$$\overline{u_iu_j} = \overline{u}_i\overline{u}_j+\tau_{ij}^R.$$
Substituting this into the filtered Navier--Stokes equation gives us
$$\frac{\partial \overline{u}_i}{\partial t} - \frac{\partial }{\partial x_j}\left(\overline{u}_i\overline{u}_j+\tau_{ij}^R \right)+\frac{\partial \overline{P}}{\partial x_i}- \nu \frac{\partial^2\overline{u}_i}{\partial x_j^2} = \overline{F}_i.$$
We make an additional simplification as follows: define the \emph{anisotropic stress tensor} as
$$\tau_{ij}^r := \tau_{ij}^R - \frac{1}{3}\delta_{ij}\tau_{kk}^R,$$
and the modified filtered pressure as
$$\overline{p} := \overline{P} - \frac{1}{3}\delta_{ij}\tau_{kk}^R.$$
Then we can express the filtered Navier--Stokes equation as
\begin{equation}\label{barsNS}
\frac{\partial \overline{u}_i}{\partial t} - \frac{\partial }{\partial x_j}\left(\overline{u}_i\overline{u}_j+\tau_{ij}^r \right)+\frac{\partial \overline{p}}{\partial x_i}- \nu \frac{\partial^2\overline{u}_i}{\partial x_j^2} = \overline{F}_i.
\end{equation}
We use this formulation because $\tau_{ij}^r$ is traceless; it contains one less independent component than $\tau_{ij}^R$. A \emph{subgrid scale model} is a model for $\tau_{ij}^r$.
\section{Two-Dimensional Subgrid models}
Assume we are in two-dimensions (i.e.\ $u_3=0$). We rewrite the \eq{barsNS}, dropping the bars for convenience.
\begin{equation}\label{filteredNS}
\frac{\partial {u}_i}{\partial t} =- \frac{\partial }{\partial x_j}\left({u}_i{u}_j +\tau_{ij}^r\right)-\frac{\partial {p}}{\partial x_i}+ \nu \frac{\partial^2u_i}{\partial x_j^2} + {F}_i.
\end{equation}

We define the vorticity of the system to be $\boldsymbol{\omega} = \boldsymbol{\nabla}\times\boldsymbol{u}$. In two-dimensions, $\omega:= \omega_3$ is the only non-zero component of the vorticity. Our goal is to obtain an equation for how $\omega$ evolves in time. We have:
$$\frac{\partial \omega}{\partial t}   =  \frac{\partial }{\partial t}\left(\frac{\partial u_2}{\partial x_1} -\frac{\partial u_1}{\partial x_2} \right)=  \frac{\partial }{\partial x_1}\frac{\partial u_2}{\partial t} -\frac{\partial }{\partial x_2}\frac{\partial u_1}{\partial t}.$$
We can use \eq{filteredNS} to obtain expressions for ${\partial u_1}/{\partial t}$ and ${\partial u_2}/{\partial t}$. This gives us:
\begin{align*}
\frac{\partial \omega}{\partial t} &= \frac{\partial }{\partial x_1}\left[- \frac{\partial }{\partial x_j}\left({u}_2{u}_j +\tau_{2j}^r\right)-\cancel{\frac{\partial {p}}{\partial x_2}}+  \nu \frac{\partial^2u_2}{\partial x_j^2}+ {F}_2\right]-\frac{\partial }{\partial x_2}\left[- \frac{\partial }{\partial x_j}\left({u}_1{u}_j +\tau_{1j}^r\right)-\cancel{\frac{\partial {p}}{\partial x_1}}+ \nu \frac{\partial^2u_1}{\partial x_j^2}+ {F}_1\right]\\
&= \frac{\partial }{\partial x_j}\left[ \frac{\partial }{\partial x_2}\left({u}_1{u}_j +\tau_{1j}^r\right) - \frac{\partial }{\partial x_1}\left({u}_2{u}_j +\tau_{2j}^r\right)\right] +\nu \frac{\partial^2}{\partial x_j^2}\left(\frac{\partial u_2}{\partial x_1} -\frac{\partial u_1}{\partial x_2}\right) +\frac{\partial F_2}{\partial x_1} -\frac{\partial F_1}{\partial x_2}\\
&= \frac{\partial^2 }{\partial x_1\partial x_2}\left({u}_1^2 +\tau_{11}^r\right) - \frac{\partial^2 }{\partial x_1^2}\left({u}_2{u}_1 +\tau_{21}^r\right)+\frac{\partial^2 }{\partial x_2^2}\left({u}_1{u}_2 +\tau_{12}^r\right) - \frac{\partial^2 }{\partial x_1\partial x_2}\left({u}_2^2+\tau_{22}^r\right) +\nu \frac{\partial^2\omega}{\partial x_j^2}+\frac{\partial F_2}{\partial x_1} -\frac{\partial F_1}{\partial x_2}.
\end{align*}
Recall that the anisotropic stress tensor is symmetric and traceless, so we have $\tau_{12}^r=\tau_{21}^r$ and $\tau_{11}^r = -\tau_{22}^r$. This gives us
$$\frac{\partial \omega}{\partial t} = \frac{\partial^2 }{\partial x_1\partial x_2}\left({u}_1^2 -{u}_2^2+2\tau_{11}^r\right) +\left(\frac{\partial^2 }{\partial x_2^2} - \frac{\partial^2 }{\partial x_1^2}\right)\left({u}_1{u}_2 +\tau_{12}^r\right) +\nu \frac{\partial^2\omega}{\partial x_j^2}+\frac{\partial F_2}{\partial x_1} -\frac{\partial F_1}{\partial x_2}.$$
Now define $A(\boldsymbol{x}, t)  := u_2^2-u_1^2- 2\tau_{11}^r$ and $B(\boldsymbol{x}, t) :=u_1u_2 + \tau_{12}^r$. Rearranging the terms, we can write the vorticity equation as:
\begin{equation}\label{vortB}
\frac{\partial \omega}{\partial t}  + \frac{\partial^2 }{\partial x_1\partial x_2}A(\boldsymbol{x}, t) + \left(\frac{\partial^2}{\partial x_1^2} - \frac{\partial^2}{\partial x_2^2}\right)B(\boldsymbol{x}, t) = \nu{\nabla}^2\omega+\frac{\partial F_2}{\partial x_1} - \frac{\partial F_1}{\partial x_2} .
\end{equation}
Notice that \eq{vortB} is quite general. It makes no assumptions about the type of model we are using. 
\section{The Smagorinsky model}
In large eddy simulations, the filtered Navier--Stokes equation includes the anisotropic residual stress tensor. This tensor requires a model (known as a subgrid scale model). The first such model is known as the Smagorinsky model.

First proposed by Joseph Smagorinsky in 1963, the model was initially used to study the dynamics of atmospheric circulation \cite{Smagorinsky1963}. While considered to be one of the simplest of subgrid scale models, it forms the basis for many more advanced  models. 

The Smagorinsky model assumes the linear turbulent viscosity model:
$$\tau_{ij}^r = -2\nu_t {S}_{ij},$$
where $\nu_t$ is the turbulent viscosity and $S_{ij}$ is the rate of strain tensor:
$${S}_{ij} := \frac{1}{2}\left( \frac{\partial {u}_i}{\partial x_j} + \frac{\partial {u}_j}{\partial x_i}\right).$$
To find $\tau_{ij}^r$, we need to find $\nu_t$. The turbulent viscosity is assumed to be given by
\begin{equation}\label{smag_nut}
\nu_t = (C_s \Delta)^2 {S},
\end{equation}
where $C_s$ is the Smagorinsky coefficient, $\Delta$ is the filter width, and ${S}$ is the characteristic filtered rate of strain:
$${S} := \left(2{S}_{ij}{S}_{ij}\right)^{1/2}.$$
This can be written out as:
$${S} = \left[\frac{1}{2}\left( \frac{\partial {u}_i}{\partial x_j} + \frac{\partial {u}_j}{\partial x_i}\right)\left( \frac{\partial {u}_i}{\partial x_j} + \frac{\partial {u}_j}{\partial x_i}\right)\right]^{1/2} = \left[\frac{1}{2}\left( \frac{\partial {u}_1}{\partial x_1}\right)^2+\left( \frac{\partial {u}_2}{\partial x_1} + \frac{\partial {u}_1}{\partial x_2}\right)^2+\frac{1}{2}\left( \frac{\partial {u}_2}{\partial x_2}\right)^2 \right]^{1/2}.$$
So the anisotropic stress tensor is given by
\begin{equation}
\tau_{ij}^r = -(C_s \Delta)^2\left[2\left( \frac{\partial {u}_1}{\partial x_1}\right)^2+4\left( \frac{\partial {u}_2}{\partial x_1} + \frac{\partial {u}_1}{\partial x_2}\right)^2+2\left( \frac{\partial {u}_2}{\partial x_2}\right)^2 \right]^{1/2}{S}_{ij}.
\end{equation}
Using the fact that the fluid is incompressible, i.e.\ $\displaystyle \frac{\partial {u}_1}{\partial x_1} = -\frac{\partial {u}_2}{\partial x_2}$, we can simplify this equation as:
\begin{equation}\label{smagTau}
\tau_{ij}^r = -2(C_s \Delta)^2\left[\left( \frac{\partial {u}_1}{\partial x_1}\right)^2+\left( \frac{\partial {u}_2}{\partial x_1} + \frac{\partial {u}_1}{\partial x_2}\right)^2 \right]^{1/2}{S}_{ij}.
\end{equation}
From \eq{vortB}, we see that we only have to compute $\tau^r_{11}$ and $\tau^r_{12}$. We have 
$${S}_{11} = \frac{\partial {u}_1}{\partial x_1},$$
and
$${S}_{12} = \frac{1}{2}\left( \frac{\partial {u}_1}{\partial x_2} + \frac{\partial {u}_2}{\partial x_1}\right).$$
So we have
\begin{equation}\label{tau11}
\tau_{11}^r= -2(C_s \Delta)^2\left[\left( \frac{\partial {u}_1}{\partial x_1}\right)^2+\left( \frac{\partial {u}_2}{\partial x_1} + \frac{\partial {u}_1}{\partial x_2}\right)^2 \right]^{1/2}\left(\frac{\partial {u}_1}{\partial x_1}\right),
\end{equation}
and
\begin{equation}\label{tau12}
\tau_{12}^r= -(C_s \Delta)^2\left[\left( \frac{\partial {u}_1}{\partial x_1}\right)^2+\left( \frac{\partial {u}_2}{\partial x_1} + \frac{\partial {u}_1}{\partial x_2}\right)^2 \right]^{1/2}\left( \frac{\partial {u}_1}{\partial x_2} + \frac{\partial {u}_2}{\partial x_1}\right).
\end{equation}
The vorticity equation, with the Smagorinsky subgrid model, is then given by \eq{vortB}, where
$$A(\boldsymbol{x}, t) = u_2^2 - u_1^2 + 4(C_s \Delta)^2\left[\left( \frac{\partial {u}_1}{\partial x_1}\right)^2+\left( \frac{\partial {u}_2}{\partial x_1} + \frac{\partial {u}_1}{\partial x_2}\right)^2 \right]^{1/2}\left(\frac{\partial {u}_1}{\partial x_1}\right),$$
and
$$B(\boldsymbol{x}, t) = u_1u_2 -(C_s \Delta)^2\left[\left( \frac{\partial {u}_1}{\partial x_1}\right)^2+\left( \frac{\partial {u}_2}{\partial x_1} + \frac{\partial {u}_1}{\partial x_2}\right)^2 \right]^{1/2}\left( \frac{\partial {u}_1}{\partial x_2} + \frac{\partial {u}_2}{\partial x_1}\right).$$
The advantage of writing the equation in this form is that in a two-dimensional pseudospectral simulation (such as \text{\tt{protodns}}), $A(\boldsymbol{x},t)$ and $B(\boldsymbol{x},t)$ simply take the place of $u_2^2 - u_1^2$ and $u_1u_2$ respectively. This, however, will require at least 2 additional inverse FFTs.
\section{The Dynamical Model}
\subsection{Theory}
The dynamical subgrid-scale model, proposed in \cite{GermanoEtAl1991}, aims to deal with a problem in turbulent-viscosity subgrid scale models. Recall that the turbulent viscosity in the Smagorinsky model is given by:
$$\nu_\tau = (C_s\Delta)^2\bar{S},$$
where $C_s$ is the Smagorinsky coefficient. What should the value of the Smagorinsky coefficient be? Unfortunately, there isn't one value that will work in all scenarios. 

The dynamical model differs from the Smagorinsky model by locally calculating the turbulent-viscosity to reflect the nature of the flow. To do this we need two filters. A grid filter $\bar{G}$ and a test filter $\tilde{G}$. We define the filter function $\tilde{\bar{G}} := \bar{G}\tilde{G}$.

Using the grid filter, the filtered Navier—Stokes eqation is given by:
$$ \frac{\partial \bar{u}_i}{\partial t} - \frac{\partial }{\partial x_j}\overline{u}_i\overline{u}_j+\frac{\partial \overline{p}}{\partial x_i}- \nu \nabla^2 \overline{u} +\frac{\partial \tau_{ij}^R}{\partial x_j}= \overline{F}_i,$$ 
where $\tau_{ij}^R = \overline{u_iu_j}-\bar{u}_i\bar{u}_j$. Using the filter $\tilde{\bar{G}}$, the filtered Navier—Stokes equation is:
$$ \frac{\partial \tilde{\bar{u}}_i}{\partial t} - \frac{\partial }{\partial x_j}\tilde{\bar{u}}_i\tilde{\bar{u}}_j+\frac{\partial \tilde{\bar{p}}}{\partial x_i}- \nu \nabla^2 \overline{u} +\frac{\partial T_{ij}^R}{\partial x_j}= \tilde{\bar{F_i}},$$ 
where $T_{ij}^R = \widetilde{\overline{u_iu_j}}-\tilde{\bar{u}}_i\tilde{\bar{u}}_j$. Now define the resolved turbulent stress as
$$\mathscr{L}_{ij}^R := \widetilde{\bar{u}_i\bar{u}_i} - \tilde{\bar{u}}_i\tilde{\bar{u}}_j.$$
We see that the resolved turbulent stress is dependent only on the filtered velocity. Furthermore, we note that we can express the resolved turbulent stress in terms of $\tau_{ij}^R$ and $T_{ij}^R$ (the so called Germano Identity \cite[p.~621]{Pope2000}):
$$\mathscr{L}_{ij}^R = T_{ij}^R - \tilde{\tau}_{ij}^R$$
In the Dynamical model, we assume:
$$\tau_{ij}^r = \tau_{ij}^R - \frac{1}{3}\delta_{ij}\tau_{kk}^R = 2C\bar{\Delta}^2|\bar{S}|\bar{S}_{ij},$$
and 
$$T_{ij}^r = T_{ij}^R - \frac{1}{3}\delta_{ij}T_{kk}^R = 2C\tilde{\bar{\Delta}}^2|\tilde{\bar{S}}|\tilde{\bar{S}}_{ij}.$$
This uses a turbulent viscosity almost identical to the Smagorinsky subgrid model \eq{smag_nut}, except here $C$ is not a squared value. Using these assumptions, we can write
\begin{equation}\label{aniso_resolved}
\mathscr{L}_{ij}^r=  2C{M}_{ij},
\end{equation}
where 
$$\mathscr{L}_{ij}^r:=\mathscr{L}_{ij}^R - \frac{1}{3}\delta_{ij}\mathscr{L}_{kk}^R,$$
and 
$${M}_{ij} :=\tilde{\bar{\Delta}}^2|\tilde{\bar{S}}|\tilde{\bar{S}}_{ij} - \bar{\Delta}^2\widetilde{|\bar{S}|\bar{S}_{ij}}.$$
Now, what we would like to do, is to find $C$ that satisfies \eq{aniso_resolved}. Unfortunately, this isn't possible. Careful inspection will reveal \eq{aniso_resolved} is actually composed of 5 independent equations; therefore, we can't choose one $unknown$ $C$ that satisfies all of them. 

A way around this problem was provided by Lilly \cite{Lilly1992}. His insight was to choose $C$ that resulted in the least square error between $\mathscr{L}_{ij}^r$ and $2C{M}_{ij}$. Define the squared error:
$$J : = \left(\mathscr{L}_{ij}^r -  2C{M}_{ij}\right)^2 = \left(\mathscr{L}_{ij}^R - \frac{1}{3}\delta_{ij}\mathscr{L}_{kk}^R -  2C{M}_{ij}\right)^2$$
Taking the derivative with respect to $C$ gives us:
$$\frac{\partial J}{\partial C} : = -4\left(\mathscr{L}_{ij}^R - \frac{1}{3}\delta_{ij}\mathscr{L}_{kk}^R -  2C{M}_{ij}\right)M_{ij}$$
We set the derivative to 0 to get the minimum:
\begin{align*}
0&=  -4\left(\mathscr{L}_{ij}^R - \frac{1}{3}\delta_{ij}\mathscr{L}_{kk}^R -  2C{M}_{ij}\right)M_{ij}\\
&=\left(\mathscr{L}_{ij}^R - \frac{1}{3}\delta_{ij}\mathscr{L}_{kk}^R\right)M_{ij} -  2C{M}_{ij}{M}_{ij} 
\end{align*}
$$\Rightarrow C = \frac{\left(\mathscr{L}_{ij}^R - \frac{1}{3}\delta_{ij}\mathscr{L}_{kk}^R\right)M_{ij}}{2{M}_{ij}{M}_{ij}}$$
Note that $\frac{1}{3}\delta_{ij}\mathscr{L}_{kk}^RM_{ij} = 0$ if $i\neq j$. If $i = j$ and the flow is incompressible, we have:
$$\frac{1}{3}\mathscr{L}_{kk}^RM_{ii} = \frac{1}{3}\mathscr{L}_{kk}^R\left(\tilde{\bar{\Delta}}^2|\tilde{\bar{S}}|\tilde{\bar{S}}_{ii} - \bar{\Delta}^2\widetilde{|\bar{S}|\bar{S}_{ii}}\right) = \frac{1}{3}\mathscr{L}_{kk}^R\left(\tilde{\bar{\Delta}}^2|\tilde{\bar{S}}|0- \bar{\Delta}^2\widetilde{|\bar{S}|0}\right) = 0$$
So this term vanishes completely, and we have:
\begin{equation}
C = \frac{\mathscr{L}_{ij}^R M_{ij}}{2{M}_{ij}{M}_{ij}}
\end{equation}
Notice that $C$ is now a function of $\boldsymbol{x}$ and $t$, and can take negative values. This corresponds to the effect known as turbulent backscatter (in which small portions of energy moves from the smaller scales to the larger scales).
\subsection{The Dynamical Model in Two-Dimensions}
Assuming we are in two dimensions, we can expand $C$ as:
$$C = \frac{1}{2}\frac{\mathscr{L}^R_{11} M_{11}+\mathscr{L}^R_{12} M_{12}+\mathscr{L}^R_{21} M_{21}+\mathscr{L}^R_{22} M_{22}}{M_{11} M_{11}+M_{12} M_{12}+M_{21} M_{21}+M_{22} M_{22}}.$$
Both $\mathscr{L}^R_{ij}$ and $M_{ij}$ are symmetric so we have
$$C = \frac{1}{2}\frac{\mathscr{L}^R_{11} M_{11}+2\mathscr{L}^R_{12} M_{12}+\mathscr{L}^R_{22} M_{22}}{M_{11}^2+2M_{12}^2+M_{22}^2}.$$
It is assumed that we are simulating the grid filtered velocity exactly. Only the test filter $\tilde{\Delta}$ needs to be applied explicitly. For this reason we drop the bars and use $\mathfrak{F}$ to denote the test filtered quantities:
$$\mathfrak{F}(u) := \tilde{u}$$
In this new notation, we write
$$\mathscr{L}^R_{ij} = \mathfrak{F}\left( u_i u_j\right) - \mathfrak{F}\left(u_i\right)\mathfrak{F}\left(u_j\right),$$
and
$$M_{ij} = \tilde{\bar{\Delta}}^2\mathfrak{F}\left(S\right)\mathfrak{F}\left({S}_{ij}\right) - \bar{\Delta}^2\mathfrak{F}\left(S{S}_{ij}\right).$$
Now we can compute:
$$\mathscr{L}^R_{11} = \mathfrak{F}\left( u_1^2\right) - \mathfrak{F}\left(u_1\right)^2,$$
$$\mathscr{L}^R_{12} = \mathfrak{F}\left( u_1u_2\right) - \mathfrak{F}\left(u_1\right)\mathfrak{F}\left(u_2\right),$$
$$\mathscr{L}^R_{22} = \mathfrak{F}\left( u_2^2\right) - \mathfrak{F}\left(u_2\right)^2,$$
and
$$M_{11} = \tilde{\bar{\Delta}}^2\mathfrak{F}\left(S\right)\mathfrak{F}\left({S}_{11}\right) - \bar{\Delta}^2\mathfrak{F}\left(S{S}_{11}\right),$$
$$M_{12} = \tilde{\bar{\Delta}}^2\mathfrak{F}\left(S\right)\mathfrak{F}\left({S}_{12}\right) - \bar{\Delta}^2\mathfrak{F}\left(S{S}_{12}\right),$$
$$M_{22} = \tilde{\bar{\Delta}}^2\mathfrak{F}\left(S\right)\mathfrak{F}\left({S}_{22}\right) - \bar{\Delta}^2\mathfrak{F}\left(S{S}_{22}\right).$$
For simplicity, define $C_n := \mathscr{L}^R_{11} M_{11}+2\mathscr{L}^R_{12} M_{12}+\mathscr{L}^R_{22} M_{22}$ and $C_d:= 2M_{11}^2+4M_{12}^2+2M_{22}^2$. This allows us to write
\begin{equation}
C = \frac{C_n}{C_d}.
\end{equation}
We deal with $C_n$ and $C_d$ separately. 
\subsection{Computing $C_n$ and $C_d$}
First, $C_n$ is given by:
\begin{multline*}
C_n = \left[\mathfrak{F}\left( u_1^2\right) - \mathfrak{F}\left(u_1\right)^2\right]\left[\tilde{\bar{\Delta}}^2\mathfrak{F}\left(S\right)\mathfrak{F}\left(\frac{\partial u_1}{\partial x_1}\right) - \bar{\Delta}^2\mathfrak{F}\left(S\frac{\partial u_1}{\partial x_1}\right)\right]\\+2\left[\mathfrak{F}\left( u_1u_2\right) - \mathfrak{F}\left(u_1\right)\mathfrak{F}\left(u_2\right)\right]\left[\tilde{\bar{\Delta}}^2\mathfrak{F}\left(S\right)\mathfrak{F}\left({S}_{12}\right) - \bar{\Delta}^2\mathfrak{F}\left(S{S}_{12}\right)\right]\\+\left[\mathfrak{F}\left( u_2^2\right) - \mathfrak{F}\left(u_2\right)^2\right]\left[\tilde{\bar{\Delta}}^2\mathfrak{F}\left(S\right)\mathfrak{F}\left(\frac{\partial u_2}{\partial x_2}\right)-\bar{\Delta}^2\mathfrak{F}\left(S\frac{\partial u_2}{\partial x_2}\right)\right]
\end{multline*}
Invoking incompressibility, we have $\frac{\partial u_1}{\partial x_1} = S_{11} = -S_{22} =-\frac{\partial u_2}{\partial x_2}$,. We can now write
\begin{multline*}
C_n = \left[\mathfrak{F}\left( u_1^2\right) - \mathfrak{F}\left(u_1\right)^2\right]\left[\tilde{\bar{\Delta}}^2\mathfrak{F}\left(S\right)\mathfrak{F}\left(\frac{\partial u_1}{\partial x_1}\right) - \bar{\Delta}^2\mathfrak{F}\left(S\frac{\partial u_1}{\partial x_1}\right)\right]\\
+2\left[\mathfrak{F}\left( u_1u_2\right) - \mathfrak{F}\left(u_1\right)\mathfrak{F}\left(u_2\right)\right]\left[\tilde{\bar{\Delta}}^2\mathfrak{F}\left(S\right)\mathfrak{F}\left({S}_{12}\right) - \bar{\Delta}^2\mathfrak{F}\left(S{S}_{12}\right)\right]\\
-\left[\mathfrak{F}\left( u_2^2\right) - \mathfrak{F}\left(u_2\right)^2\right]\left[\tilde{\bar{\Delta}}^2\mathfrak{F}\left(S\right)\mathfrak{F}\left(\frac{\partial u_1}{\partial x_1}\right)-\bar{\Delta}^2\mathfrak{F}\left(S\frac{\partial u_1}{\partial x_1}\right)\right],
\end{multline*}
which can be simplified as
\begin{multline}\label{C_n}
C_n = \left[\tilde{\bar{\Delta}}^2\mathfrak{F}\left(S\right)\mathfrak{F}\left(\frac{\partial u_1}{\partial x_1}\right) - \bar{\Delta}^2\mathfrak{F}\left(S\frac{\partial u_1}{\partial x_1}\right)\right]\left[\mathfrak{F}\left( u_1^2\right) + \mathfrak{F}\left(u_2\right)^2 - \mathfrak{F}\left(u_1\right)^2 - \mathfrak{F}\left( u_2^2\right) \right]\\
+2\left[\mathfrak{F}\left( u_1u_2\right) - \mathfrak{F}\left(u_1\right)\mathfrak{F}\left(u_2\right)\right]\left[\tilde{\bar{\Delta}}^2\mathfrak{F}\left(S\right)\mathfrak{F}\left({S}_{12}\right) - \bar{\Delta}^2\mathfrak{F}\left(S{S}_{12}\right)\right].
\end{multline}
Now $C_d$ is given by:
\begin{multline*}
C_d = 2\left[\tilde{\bar{\Delta}}^2\mathfrak{F}\left(S\right)\mathfrak{F}\left({S}_{11}\right) - \bar{\Delta}^2\mathfrak{F}\left(S{S}_{11}\right)\right]^2\\
+4\left[\tilde{\bar{\Delta}}^2\mathfrak{F}\left(S\right)\mathfrak{F}\left({S}_{12}\right) - \bar{\Delta}^2\mathfrak{F}\left(S{S}_{12}\right)\right]^2\\
+2\left[\tilde{\bar{\Delta}}^2\mathfrak{F}\left(S\right)\mathfrak{F}\left({S}_{22}\right) - \bar{\Delta}^2\mathfrak{F}\left(S{S}_{22}\right)\right]^2.
\end{multline*}
Again, invoking incompressibility, gives us
\begin{multline*}
C_d = 2\left[\tilde{\bar{\Delta}}^2\mathfrak{F}\left(S\right)\mathfrak{F}\left({S}_{11}\right) - \bar{\Delta}^2\mathfrak{F}\left(S{S}_{11}\right)\right]^2\\
+4\left[\tilde{\bar{\Delta}}^2\mathfrak{F}\left(S\right)\mathfrak{F}\left({S}_{12}\right) - \bar{\Delta}^2\mathfrak{F}\left(S{S}_{12}\right)\right]^2\\
+2\left[\tilde{\bar{\Delta}}^2\mathfrak{F}\left(S\right)\mathfrak{F}\left({S}_{11}\right) - \bar{\Delta}^2\mathfrak{F}\left(S{S}_{11}\right)\right]^2,
\end{multline*}
which results in the expression for $C_d$:
\begin{equation}\label{C_d}
C_d = 4\left\{\left[\tilde{\bar{\Delta}}^2\mathfrak{F}\left(S\right)\mathfrak{F}\left({S}_{11}\right) - \bar{\Delta}^2\mathfrak{F}\left(S{S}_{11}\right)\right]^2+\left[\tilde{\bar{\Delta}}^2\mathfrak{F}\left(S\right)\mathfrak{F}\left({S}_{12}\right) - \bar{\Delta}^2\mathfrak{F}\left(S{S}_{12}\right)\right]^2\right\}.
\end{equation}
This value can then be used to model the anisotropic residule stress tensor:
$$\tau_{ij}^r = 2C\bar{\Delta}^2{S}{S}_{ij}$$.
In the dynamic model, we then evolve the following equation:
$$\frac{\partial \omega}{\partial t}  + \frac{\partial^2 }{\partial x_1\partial x_2}A(\boldsymbol{x}, t) + \left(\frac{\partial^2}{\partial x_1^2} - \frac{\partial^2}{\partial x_2^2}\right)B(\boldsymbol{x}, t) = \nu{\nabla}^2\omega+\frac{\partial F_2}{\partial x_1} - \frac{\partial F_1}{\partial x_2}.$$
Where 
$$A(\boldsymbol{x}, t) = u_2^2 - u_1^2 + 4C(\bar{\Delta})^2\left[\left( \frac{\partial {u}_1}{\partial x_1}\right)^2+\left( \frac{\partial {u}_2}{\partial x_1} + \frac{\partial {u}_1}{\partial x_2}\right)^2 \right]^{1/2}\left(\frac{\partial {u}_1}{\partial x_1}\right),$$
and
$$B(\boldsymbol{x}, t) = u_1u_2 -C(\bar{\Delta})^2\left[\left( \frac{\partial {u}_1}{\partial x_1}\right)^2+\left( \frac{\partial {u}_2}{\partial x_1} + \frac{\partial {u}_1}{\partial x_2}\right)^2 \right]^{1/2}\left( \frac{\partial {u}_1}{\partial x_2} + \frac{\partial {u}_2}{\partial x_1}\right).$$
Notice that $C$ is now a function of $\boldsymbol{x}$ and $t$, and can take negative values. This corresponds to the effect known as turbulent backscatter (in which small portions of energy moves from the smaller scales to the larger scales).
\subsection{Algorithm}
An issue with using the dynamical model in a pseudospectral simulation (such as \text{\tt{protodns}}) is that a filter is best applied in Fourier space (as it is a convolution in physical space), but the non-linear terms are best computed in physical space. In a regular pseudospectral simulation, one transforms to physical space once to compute the non-linear terms and then back to Fourier space to compute the next time step. 

Now, one will have to transform to physical space to compute non-linear terms, transform to Fourier space to apply the filters, transform back to physical space to compute non-linear terms using these filtered quantities, then back into Fourier space to advance to the next time step.

What follows is a possible algorithm for computing the nonlinear terms $A(\boldsymbol{x}, t)$ and $B(\boldsymbol{x},t)$ at each time step. We use $\mathcal{F}_{\boldsymbol{k}}[{f}]$ to denote the Fourier transform of a function $f$.
\begin{enumerate}
\item In fourier space, compute $\mathcal{F}_{\boldsymbol{k}}\left[{\mathfrak{F}\left({u}_{1}\right)}\right]$ and $\mathcal{F}_{\boldsymbol{k}}\left[{\mathfrak{F}\left({u}_{2}\right)}\right]$.
\item Using \textbf{eight} inverse FFTs, trasform the following:
$$\mathcal{F}_{\boldsymbol{k}}[{u}_{2}] \mapsto {u}_{1}$$
$$\mathcal{F}_{\boldsymbol{k}}[{u}_{2}] \mapsto {u}_{2}$$
$$\mathcal{F}_{\boldsymbol{k}}\left[{\mathfrak{F}\left({u}_{1}\right)}\right] \mapsto \mathfrak{F}\left({u}_{1}\right)$$
$$\mathcal{F}_{\boldsymbol{k}}\left[{\mathfrak{F}\left({u}_{2}\right)}\right]\mapsto \mathfrak{F}\left({u}_{2}\right)$$
$$\mathcal{F}_{\boldsymbol{k}}\left[\frac{\partial u_{1}}{\partial x_1}\right] \mapsto \frac{\partial {u}_{1}}{\partial x_1}$$
$$\mathcal{F}_{\boldsymbol{k}}\left[\frac{\partial u_{2}}{\partial x_1}+\frac{\partial u_{1}}{\partial x_2}\right] \mapsto  \frac{\partial {u}_{1}}{\partial x_2}+\frac{\partial {u}_{2}}{\partial x_1}$$
$$\mathcal{F}_{\boldsymbol{k}}\left[\mathfrak{F}\left(\frac{\partial u_{1}}{\partial x_1}\right)\right] \mapsto \mathfrak{F}\left(\frac{\partial u_{1}}{\partial x_1}\right)$$
$$\mathcal{F}_{\boldsymbol{k}}\left[\mathfrak{F}\left(\frac{\partial u_{2}}{\partial x_1}+\frac{\partial u_{1}}{\partial x_2}\right)\right] \mapsto  \mathfrak{F}\left(\frac{\partial u_{2}}{\partial x_1}+\frac{\partial u_{1}}{\partial x_2}\right)$$
\item Compute $u_1^2$, $u_2^2$, $u_1u_2$ in physical space.
\item Compute $S$ in physical space which is given by
$$S = \left[\left( \frac{\partial {u}_1}{\partial x_1}\right)^2+\left( \frac{\partial {u}_2}{\partial x_1} + \frac{\partial {u}_1}{\partial x_2}\right)^2\right]^{1/2}.$$
\item In physical space compute $S \frac{\partial u_1}{\partial x_1}$ and $S S_{12}$ where
$$S_{12} = \frac{1}{2}\left(\frac{\partial {u}_{1}}{\partial x_2}+\frac{\partial {u}_{2}}{\partial x_1}\right).$$
\item Using \textbf{six} FFTs, transform the following:
$$u_1^2\mapsto \mathcal{F}_{\boldsymbol{k}}\left[{u_1^2}\right]$$
$$u_2^2\mapsto \mathcal{F}_{\boldsymbol{k}}\left[{u_2^2}\right]$$
$$u_1u_2\mapsto \mathcal{F}_{\boldsymbol{k}}\left[{u_1u_2}\right]$$
$$S\mapsto \mathcal{F}_{\boldsymbol{k}}\left[{S}\right]$$
$$S \frac{\partial u_1}{\partial x_1}\mapsto \mathcal{F}_{\boldsymbol{k}}\left[{S\frac{\partial u_1}{\partial x_1}}\right]$$
$$S S_{12}\mapsto \mathcal{F}_{\boldsymbol{k}}\left[{S S_{12}}\right]$$
\item Compute $\mathcal{F}_{\boldsymbol{k}}\left[{\mathfrak{F}\left(u_1^2\right)}\right]$, $\mathcal{F}_{\boldsymbol{k}}\left[{\mathfrak{F}\left(u_2^2\right)}\right]$, $\mathcal{F}_{\boldsymbol{k}}\left[{\mathfrak{F}\left(u_1u_2\right)}\right]$, $\mathcal{F}_{\boldsymbol{k}}\left[{\mathfrak{F}\left(S\right)}\right]$, $\mathcal{F}_{\boldsymbol{k}}\left[{\mathfrak{F}\left(S  \frac{\partial u_1}{\partial x_1}\right)}\right]$, $\mathcal{F}_{\boldsymbol{k}}\left[{\mathfrak{F}\left(S  S_{12}\right)}\right]$
\item Using \textbf{six} inverse FFTs, transform the following:
$$\mathcal{F}_{\boldsymbol{k}}\left[{\mathfrak{F}\left(u_1^2\right)}\right] \mapsto \mathfrak{F}\left(u_1^2\right)$$
$$\mathcal{F}_{\boldsymbol{k}}\left[{\mathfrak{F}\left(u_2^2\right)}\right] \mapsto \mathfrak{F}\left(u_2^2\right)$$
$$\mathcal{F}_{\boldsymbol{k}}\left[{\mathfrak{F}\left(u_1u_2\right)}\right] \mapsto \mathfrak{F}\left(u_1u_2\right)$$
$$\mathcal{F}_{\boldsymbol{k}}\left[{\mathfrak{F}\left(S\right)}\right] \mapsto \mathfrak{F}\left(S\right)$$
$$\mathcal{F}_{\boldsymbol{k}}\left[{\mathfrak{F}\left(S\frac{\partial u_1}{\partial x_1}\right)}\right] \mapsto \mathfrak{F}\left(S\frac{\partial u_1}{\partial x_1}\right)$$
$$\mathcal{F}_{\boldsymbol{k}}\left[{\mathfrak{F}\left(SS_{12}\right)}\right] \mapsto \mathfrak{F}\left(SS_{12}\right)$$
\item Compute $C = C_n/C_d$ where $C_n$ is given by \eq{C_n} and $C_d$ is given by \eq{C_d}.
\item Compute
$$A(\boldsymbol{x}, t) = u_2^2 - u_1^2 + 4C(\bar{\Delta})^2\left[\left( \frac{\partial {u}_1}{\partial x_1}\right)^2+\left( \frac{\partial {u}_2}{\partial x_1} + \frac{\partial {u}_1}{\partial x_2}\right)^2 \right]^{1/2}\left(\frac{\partial {u}_1}{\partial x_1}\right),$$
and 
$$B(\boldsymbol{x}, t) = u_1u_2 -C(\bar{\Delta})^2\left[\left( \frac{\partial {u}_1}{\partial x_1}\right)^2+\left( \frac{\partial {u}_2}{\partial x_1} + \frac{\partial {u}_1}{\partial x_2}\right)^2 \right]^{1/2}\left( \frac{\partial {u}_1}{\partial x_2} + \frac{\partial {u}_2}{\partial x_1}\right).$$
\item Using \textbf{two} FFTs, trasform the following:
$$A(\boldsymbol{x}, t) \mapsto \mathcal{F}_{\boldsymbol{k}}\left[A(\boldsymbol{x}, t) \right]$$
$$B(\boldsymbol{x}, t) \mapsto \mathcal{F}_{\boldsymbol{k}}\left[B(\boldsymbol{x}, t) \right]$$
\end{enumerate}

In this algorithm, we use a total of \textbf{22} FFTs at each time step. This almost 4 times the number needed for the smagorinsky model. It is possible, however, that this number could be reduced.
\bibliographystyle{plain}
\clearpage
\bibliography{refs}
\end{document}
