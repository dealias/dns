\documentclass[11pt,a4paper]{article}
\usepackage{amsmath}
\usepackage{amsfonts}
\usepackage{cancel}
\usepackage{amssymb}
\usepackage{graphicx}
\usepackage[left=2cm,right=2cm,top=2cm,bottom=2cm]{geometry}
\author{Noel Murasko}
\title{Two-Dimensional Pseudospectral Large Eddy Simulation}
\date{\today}
\begin{document}
\maketitle
\noindent 
\section{Two-Dimensional Subgrid models}
We can write the filtered Navier--Stokes equation (for an incompressible fluid) as
$$\frac{\partial {u}_i}{\partial t} =- \frac{\partial }{\partial x_j}({u}_i{u}_j)-\frac{\partial {p}}{\partial x_i}+ \nu \frac{\partial^2u_i}{\partial x_j^2} -\frac{\partial \tau_{ij}^r}{\partial x_j} + {F}_i,$$
Here $u_i$ denotes the filtered velocity component in the direction of $x_i$, and $p$ denotes the modified\footnote{The filtered equations can be derived in terms of the residual stress tensor $\tau_{ij}^R$. However, we have written them in terms of $\tau_{ij}^r := \tau^R_{ij} - \frac{1}{3}\delta_{ij}\tau^R_{kk}$. We use $\tau_{ij}^r$ in the equation and absorb $\frac{1}{3}\delta_{ij}\tau^R_{kk}$ into the pressure term. Thus $p$ in equation \ref{filteredNS} is not the true filtered pressure but a modified scalar related to it.} filtered pressure. $\tau^r_{ij}$ is the anisotropic residule stress tensor and is a model for the unresolved terms. 

An immediate simplification one can make is to combine the first and fourth terms on the RHS. Then we have
\begin{equation}\label{filteredNS}
\frac{\partial {u}_i}{\partial t} =- \frac{\partial }{\partial x_j}\left({u}_i{u}_j +\tau_{ij}^r\right)-\frac{\partial {p}}{\partial x_i}+ \nu \frac{\partial^2u_i}{\partial x_j^2} + {F}_i.
\end{equation}

We define the vorticity of the system to be $\boldsymbol{\omega} = \boldsymbol{\nabla}\times\boldsymbol{u}$. Assume we are in 2D (i.e. $u_3=0$). Then $\omega:= \omega_3$ is the only non-zero component of the vorticity. Our goal is to obtain an equation for how $\omega$ evolves in time. We have:
$$\frac{\partial \omega}{\partial t} = \frac{\partial }{\partial t}\left( \boldsymbol{\nabla}\times\boldsymbol{u}\right)_{3} =  \frac{\partial }{\partial t}\left(\frac{\partial u_2}{\partial x_1} -\frac{\partial u_1}{\partial x_2} \right)=  \frac{\partial }{\partial x_1}\frac{\partial u_2}{\partial t} -\frac{\partial }{\partial x_2}\frac{\partial u_1}{\partial t}$$
We can use equation \ref{filteredNS} to obtain equations for ${\partial u_1}/{\partial t}$ and ${\partial u_2}/{\partial t}$:
\begin{align*}
\frac{\partial \omega}{\partial t} &= \frac{\partial }{\partial x_1}\left[- \frac{\partial }{\partial x_j}\left({u}_2{u}_j +\tau_{2j}^r\right)-\cancel{\frac{\partial {p}}{\partial x_2}}+  \nu \frac{\partial^2u_2}{\partial x_j^2}+ {F}_2\right]-\frac{\partial }{\partial x_2}\left[- \frac{\partial }{\partial x_j}\left({u}_1{u}_j +\tau_{1j}^r\right)-\cancel{\frac{\partial {p}}{\partial x_1}}+ \nu \frac{\partial^2u_1}{\partial x_j^2}+ {F}_1\right]\\
&= \frac{\partial }{\partial x_j}\left[ \frac{\partial }{\partial x_2}\left({u}_1{u}_j +\tau_{1j}^r\right) - \frac{\partial }{\partial x_1}\left({u}_2{u}_j +\tau_{2j}^r\right)\right] +\nu \frac{\partial^2}{\partial x_j^2}\left(\frac{\partial u_2}{\partial x_1} -\frac{\partial u_1}{\partial x_2}\right) +\frac{\partial F_2}{\partial x_1} -\frac{\partial F_1}{\partial x_2}\\
&= \frac{\partial^2 }{\partial x_1\partial x_2}\left({u}_1^2 +\tau_{11}^r\right) - \frac{\partial^2 }{\partial x_1^2}\left({u}_2{u}_1 +\tau_{21}^r\right)+\frac{\partial^2 }{\partial x_2^2}\left({u}_1{u}_2 +\tau_{12}^r\right) - \frac{\partial^2 }{\partial x_1\partial x_2}\left({u}_2^2+\tau_{22}^r\right) +\nu \frac{\partial^2\omega}{\partial x_j^2}+\frac{\partial F_2}{\partial x_1} -\frac{\partial F_1}{\partial x_2}.
\end{align*}
Recall that the anisotropic stress tensor is symmetric and traceless so we have $\tau_{12}^r=\tau_{21}^r$ and $\tau_{11}^r = -\tau_{22}^r$. This gives us
$$\frac{\partial \omega}{\partial t} = \frac{\partial^2 }{\partial x_1\partial x_2}\left({u}_1^2 -{u}_2^2+2\tau_{11}^r\right) +\left(\frac{\partial^2 }{\partial x_2^2} - \frac{\partial^2 }{\partial x_1^2}\right)\left({u}_1{u}_2 +\tau_{12}^r\right) +\nu \frac{\partial^2\omega}{\partial x_j^2}+\frac{\partial F_2}{\partial x_1} -\frac{\partial F_1}{\partial x_2}.$$
Now define $A(\boldsymbol{x}, t) :=u_1u_2 + \tau_{12}^r$ and $B(\boldsymbol{x}, t)  := u_2^2-u_1^2- 2\tau_{11}^r$. Rearranging the terms, we can write the vorticity Equation as:
\begin{equation}\label{vortB}
\frac{\partial \omega}{\partial t}  + \left(\frac{\partial^2}{\partial x_1^2} - \frac{\partial^2}{\partial x_2^2}\right)A(\boldsymbol{x}, t) + \frac{\partial^2 }{\partial x_1\partial x_2}B(\boldsymbol{x}, t) = \nu{\nabla}^2\omega+\frac{\partial F_2}{\partial x_1} - \frac{\partial F_1}{\partial x_2} .
\end{equation}
Notice that equation \ref{vortB} is quite general. It makes no assumptions about the type of model we are using. 
\section{The Smagorinsky model}
In large eddy simulations, the filtered Navier--Stokes equation includes the anisotropic residual stress tensor. This tensor requires a model (known as a subgrid scale model). The first such model is known as the Smagorinsky model.

First proposed by Joseph Smagorinsky in 1963, the model was initially used to study the dynamics of atmospheric circulation. While considered to be one of the simplest of subgrid scale models, it forms the basis for many more advanced  models. 

The Smagorinsky model assumes the linear turbulent viscosity model:
$$\tau_{ij}^r = -2\nu_t {S}_{ij},$$
where $\nu_t$ is the turbulent viscosity and $S_{ij}$ is the rate of strain tensor:
$${S}_{ij} := \frac{1}{2}\left( \frac{\partial {u}_i}{\partial x_j} + \frac{\partial {u}_j}{\partial x_i}\right).$$
To find $\tau_{ij}^r$, we need to find $\nu_t$. The turbulent viscosity is assumed to be given by
$$\nu_t = (C_s \Delta)^2 {S},$$
Where $C_s$ is the Smagorinsky coefficient, $\Delta$ is the filter width, and where ${S}$ is the characteristic filtered rate of strain:
$${S} := \left(2{S}_{ij}{S}_{ij}\right)^{1/2}$$
Which we can write out as:
$${S} = \left[\frac{1}{2}\left( \frac{\partial {u}_i}{\partial x_j} + \frac{\partial {u}_j}{\partial x_i}\right)\left( \frac{\partial {u}_i}{\partial x_j} + \frac{\partial {u}_j}{\partial x_i}\right)\right]^{1/2} = \left[\frac{1}{2}\left( \frac{\partial {u}_1}{\partial x_1}\right)^2+\left( \frac{\partial {u}_2}{\partial x_1} + \frac{\partial {u}_1}{\partial x_2}\right)^2+\frac{1}{2}\left( \frac{\partial {u}_2}{\partial x_2}\right)^2 \right]^{1/2}.$$
So the anisotropic stress tensor is given by
\begin{equation}
\tau_{ij}^r = -(C_s \Delta)^2\left[2\left( \frac{\partial {u}_1}{\partial x_1}\right)^2+4\left( \frac{\partial {u}_2}{\partial x_1} + \frac{\partial {u}_1}{\partial x_2}\right)^2+2\left( \frac{\partial {u}_2}{\partial x_2}\right)^2 \right]^{1/2}{S}_{ij}.
\end{equation}
If the fluid is incompressible, i.e. $\frac{\partial {u}_1}{\partial x_1} = -\frac{\partial {u}_2}{\partial x_2}$, we can simplify this equations
\begin{equation}\label{smagTau}
\tau_{ij}^r = -2(C_s \Delta)^2\left[\left( \frac{\partial {u}_1}{\partial x_1}\right)^2+\left( \frac{\partial {u}_2}{\partial x_1} + \frac{\partial {u}_1}{\partial x_2}\right)^2 \right]^{1/2}{S}_{ij}.
\end{equation}
\section{Smagorinsky in Two-Dimensions}
Now we apply the Smagorinsky model. From equation \ref{vortB}, we see that we only have to compute $\tau^r_{11}$ and $\tau^r_{12}$. We have 
$${S}_{11} = \frac{1}{2}\left( \frac{\partial {u}_1}{\partial x_1} + \frac{\partial {u}_1}{\partial x_1}\right) = \frac{\partial {u}_1}{\partial x_1}.$$
$${S}_{12} = \frac{1}{2}\left( \frac{\partial {u}_1}{\partial x_2} + \frac{\partial {u}_2}{\partial x_1}\right)$$
So we have:
\begin{equation}\label{tau12}
\tau_{12}^r= -(C_s \Delta)^2\left[\left( \frac{\partial {u}_1}{\partial x_1}\right)^2+\left( \frac{\partial {u}_2}{\partial x_1} + \frac{\partial {u}_1}{\partial x_2}\right)^2 \right]^{1/2}\left( \frac{\partial {u}_1}{\partial x_2} + \frac{\partial {u}_2}{\partial x_1}\right).
\end{equation}
and
\begin{equation}\label{tau11}
\tau_{11}^r= -2(C_s \Delta)^2\left[\left( \frac{\partial {u}_1}{\partial x_1}\right)^2+\left( \frac{\partial {u}_2}{\partial x_1} + \frac{\partial {u}_1}{\partial x_2}\right)^2 \right]^{1/2}\left(\frac{\partial {u}_1}{\partial x_1}\right).
\end{equation}
The vorticity equation with the Smagorinsky subgrid model, is then given by \ref{vortB}, where
$$A(\boldsymbol{x}, t) = u_1u_2 -(C_s \Delta)^2\left[\left( \frac{\partial {u}_1}{\partial x_1}\right)^2+\left( \frac{\partial {u}_2}{\partial x_1} + \frac{\partial {u}_1}{\partial x_2}\right)^2 \right]^{1/2}\left( \frac{\partial {u}_1}{\partial x_2} + \frac{\partial {u}_2}{\partial x_1}\right).$$
and 
$$B(\boldsymbol{x}, t) = u_2^2 - u_1^2 + 4(C_s \Delta)^2\left[\left( \frac{\partial {u}_1}{\partial x_1}\right)^2+\left( \frac{\partial {u}_2}{\partial x_1} + \frac{\partial {u}_1}{\partial x_2}\right)^2 \right]^{1/2}\left(\frac{\partial {u}_1}{\partial x_1}\right).$$
The advantage of writing the equation in this form, is that in a two-dimensional pseudospectral simulation (such as \text{\tt{protodns}}), $B_1(\boldsymbol{x},t)$ and $B_2(\boldsymbol{x},t)$ simply take the place of $u_1u_2$ and $u_2^2 - u_1^2$ respectively. This, however, will require at least 2 additional inverse FFTs.
\end{document}
Which gives us:
\begin{equation}\label{tau22-tau11}
\tau_{22}^r - \tau_{11}^r= -(C_s \Delta)^2\left[2\left( \frac{\partial {u}_1}{\partial x_1}\right)^2+4\left( \frac{\partial {u}_2}{\partial x_1} + \frac{\partial {u}_1}{\partial x_2}\right)^2+2\left( \frac{\partial {u}_2}{\partial x_2}\right)^2 \right]^{1/2}\left(\frac{\partial {u}_2}{\partial x_2} - \frac{\partial {u}_1}{\partial x_1}\right).
\end{equation}
